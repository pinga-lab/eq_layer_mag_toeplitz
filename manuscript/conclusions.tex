\section{Conclusions}

In this work, we were able to develop a fast equivalent layer technique for processing magnetic data with the method of Conjugate Gradient Least Square using the convolutional equivalent layer theory presented in \cite{siqueira-etal2017} to obtain similar results. The sensitivity matrix of the magnetic equivalent layer carries the structure of BTTB matrices, which means a very low computational cost matrix-vector product and also the possibility to store only the first column of the matrix BCCB. In this work we propose a method to calculate the first six columns of the second derivatives matrices to embbed them into BCCB and then finally arrive in what would be the first column of the BCCB matrix embbeded from the sensitivity matrix.

Synthetic tests showed similar results estimating the physical property using a classical approach to solve a linear system and our method using the CGLS combined with the BTTB matrix-vector product. The difference in time, however, is noticeable: $2.04$ seconds using the classical approach and $0.083$ seconds using our approach. This difference was obtained with a mid-size mesh of $80 \times 80$ points, greater results can be obtained if more observation points are used.

Real data test were also conducted in the region of Carajás, Pará, Brazil. With a regular grid of 2,500,000 observation points, store the full sensitivity matrix it would be necessary 45.47 Terabytes of RAM.  However, taking advantage symmetric or skew-symmetric matrices structures, it is possible to reconstruct the whole sensitivity matrix using only 114.44 Megabytes.
Using 50 iterations of the CGLS method took $15.5$ seconds and very good results of property estimative were obtained. Also the upward continuation transformation was successfull with good results and taking only $0.49$ seconds.