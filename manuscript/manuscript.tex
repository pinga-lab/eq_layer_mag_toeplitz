%\documentclass[paper,twocolumn,twoside]{geophysics}
\documentclass[manuscript,noblind]{geophysics}
%\documentclass[manuscript]{geophysics}

% An example of defining macros
\newcommand{\rs}[1]{\mathstrut\mbox{\scriptsize\rm #1}}
\newcommand{\rr}[1]{\mbox{\rm #1}}

% Extra packages
\usepackage{amsmath}
%\usepackage[]{algorithm2e}
\usepackage{algorithm}
\usepackage{bm}
\usepackage[hyphens,spaces]{url}
\usepackage[pdftex,colorlinks=true]{hyperref}
\hypersetup{
	allcolors=black,
}
\usepackage{lipsum}
\usepackage[table]{xcolor}

\begin{document}

\title{Convolutional equivalent layer for magnetic data processing}

\renewcommand{\thefootnote}{\fnsymbol{footnote}} 

\ms{GEO-XXXX} % manuscript number

\address{
\footnotemark[2]Observat\'{o}rio Nacional, Department of Geophysics, Rio de Janeiro, Brazil\\
\footnotemark[1] Corresponding author: diego.takahashi@gmail.com
}
\author{Diego Takahashi\footnotemark[2]\footnotemark[1], Vanderlei C. Oliveira{ }Jr.\footnotemark[2] and 
Val{\'e}ria C. F. Barbosa\footnotemark[2]}

%\footer{Example}
\lefthead{Takahashi, Oliveira{ }Jr. \& Barbosa}
\righthead{Magnetic convolutional equivalent layer}

\maketitle

% Main body
\begin{abstract}

We present a fast equivalent layer method for processing large-scale magnetic data. 
We demonstrate that the sensitivity matrix associated with an equivalent layer
of dipoles can be arranged to a Block-Toeplitz Toeplitz-Block (BTTB) structure for the 
case in which observations and dipoles are aligned on a 
horizontal and regularly-spaced grid.
The product of a BTTB matrix and an arbitrary vector represents a discrete 
convolution and can be efficiently computed via 2D Fast Fourier Transform.
In this case, the matrix-vector product uses only the elements forming the first column
of the BTTB matrix, saving computational time and memory. 
Our convolutional equivalent layer method uses this approach to compute 
the matrix-vector products in the iterative conjugate gradient algorithm with the purpose 
of estimating the physical-property distribution over the equivalent layer for 
large data sets.
Synthetic tests with a mid-size $100 \times 50$ grid of total-field anomaly data
show a decrease of $\approx 10^4$ in floating-point operations and $\approx 25\times$ 
in computation runtime of our method compared to the classical approach of solving
the least-squares normal equations via Cholesky decomposition. 
Faster results are obtained for millions of data, showing drastic decreases in RAM usage
and runtime, allowing to perform magnetic data processing of large data sets on regular 
desktop computers. 
Our results also show that, compared to the classical Fourier approach, the magnetic
data processing with our method requires similar computation time, but produces significantly 
smaller border effects without using any padding scheme and also is more robust to 
deal with data on irregularly spaced points or on undulating observation surfaces.
A test with $1,310,000$ irregularly spaced field data over the Caraj{\'a}s Province, Brazil, 
confirms the efficiency of our method by taking $\approx 385.56$ seconds to estimate the physical-property
distribution over the equivalent layer and $\approx 2.64$ seconds to compute the upward 
continuation.

\end{abstract}
\section{Introduction}

Throughout the years many authors presented solutions to inverse problems in potential methods using magnetic data. The most common approaches are the works that try to parameterize the geometric source \citep{bott1960use,pilkington1986determination,bhattacharyya1980generalized} and others tried to estimate the susceptibility or the magnetization as the parameter \citep{parker1974inversion,cribb1976application,li19963,liu20173d}. With advances in computation power, the equivalent layer is an inversion technique that is becoming more present in data processing nowadays, despite its use since 1960 in geophysics literature \citep{danes1961structure,bott1967solution,dampney1969}. The equations deductions of the equivalent layer as a solution of the Laplace's equation in the region above the source was first presented by \cite{kellogg1929} and detailed explanations can also be found in \cite{blakely1996}. In magnetic data procesing, some authors explored this technique for calculating the first and second vertical derivatives fields \citep{emilia1973}, reduction to the pole \citep{silva1986,oliveirajr-etal2013,li2014using}, upward/downward continuations \citep{hansen-miyazaki1984,li-oldenburg2010} and total magnetic induction vector components calculation \citep{sun2019constrained}.

Together with the rise in computational processing, some works tried new implementations to increase the efficiency of the equivalent layer. In \cite{leao-silva1989} the authors used a shifting window over the layer, increasing the number of linear systems to be solved, but decreasing the size of it. In \cite{li-oldenburg2010} a method of transforming the full sensitivity matrix into a sparse one by using wavelets was another approach. \cite{oliveirajr-etal2013} also proposed a shifting window method, but instead of directly calculating the equivalent sources physical properties, they estimated the coefficients of a bivariate polynomial funtion representing the sources. An iterative method for gravimetric data was proposed by \cite{siqueira-etal2017}, where the mass distribution over the layer is corrected at each iteration by the residuals of observed and estimated data.

In \cite{takahashi2020convolutional}, the authors combined the fast equivalent source technique presented by \cite{siqueira-etal2017} with the concept of symmetric block-Toeplitz Toeplitz-block (BTTB) matrices to introduce the convolutional equivalent layer for gravimetric data technique. When the sensitivity matrix of the linear system, required to solve the gravimetric equivalent layer, is calculated on a regular spaced grid of dataset with constant height and each equivalent source is exactly beneath each observed data point, the BTTB structure appears. This work showed an decrease in the order of $10^4$ in floating-point operations needed to estimate the equivalent sources.

In potential field methods, the properties of Toeplitz system have been used for downward-continuation \citep{zhang-etal2016} and for 3D gravity-data inversion using a 2D multilayer model \citep{zhang-wong2015}. More recently in \cite{hogue2020tutorial} the authors provided an overview on modeling the gravity and magnetic kernels using the BTTB structures and \cite{renaut2020fast} used for inversion of both gravity and magnetic data to recover sparse subsurface structures.

A wide variety of applications in mathematics and engineering that fall into Toeplitz systems propelled the development of a large variety of  methods for solving them. Direct methods were conceived by \cite{levinson1946} and by \cite{trench1964}. Currently, the iterative method of conjugate gradient is used in most cases, in \cite{chan-jin2007} the authors presented an introduction on the topic for 1D data structures of Toeplitz matrices and also for 2D data structures, which they called block-Toeplitz Toeplitz-block matrices. In both cases, the solving strategy is to embbed the Toeplitz/BTTB matrix into a Circulant/Block-Circulant Circulant-Block matrix, calculate its eigenvalues by a 1D or 2D fast Fourier transform of its first column, respectively and carry the matrix-vector product between kernel and parameters at each iteration of the conjugate gradient method in a very fast manner.

In this work, the convolutional equivalent layer using the block-Toeplitz Toeplitz-block idea, presented in \cite{takahashi2020convolutional}, will be used to solve the linear system required to estimate the physical property that produces a magnetic field on regular grids. Despite the non-existence of a fast equivalent source technique solver as with gravity data, it is possible to achive very fast solutions using a conjugate gradient algorithm combined with the fast Fourier transform. We present a novel method of exploring the symmetric structures of the second derivatives of the inverse of the distance contained in the magnetic kernel, to keep the memory RAM usage to the minimal by using only one equivalent source to carry the calculations of the forward problem. We also show tests of the magnetic convolutional equivalent layer when irregular grids are used. The convergence of the conjugate gradient maintains in an acceptable level even under strong deviations of the data acquisition line. This shows that a fast and robust technique is possible when processing large amounts of magnetic data using the equivalent layer.
\section{Methodology}

%======================================================================================
\subsection{Classical equivalent layer for magnetic data}
%======================================================================================

Let $\mathbf{d}^{o}$ be the $N \times 1$ observed data vector, whose $i$th element 
is the total-field anomaly $d^{o}_{i}$ produced by arbitrarily magnetized sources
at the position $(x_{i}, y_{i}, z_{i})$, $i =  1, \dots, N $, 
of a right-handed Cartesian coordinate system with $x$-, $y$- and $z$-axis 
pointing to north, east and down, respectively.
We consider that the total-field anomaly data $d^{o}_{i}$ represent the discrete
values of a harmonic function. Besides, we consider that the main geomagnetic field 
direction at the study area can be defined by the unit vector
\begin{equation}
\hat{\mathbf{F}} = \begin{bmatrix}
F_x \\
F_y \\
F_z
\end{bmatrix} =
\begin{bmatrix}
\cos(I_{0}) \, \cos(D_{0}) \\
\cos(I_{0}) \, \sin(D_{0}) \\
\sin(I_{0})
\end{bmatrix} \: ,
\label{eq:unit_vector_F}
\end{equation}
with constant inclination $I_{0}$ and declination $D_{0}$.
In this case, $d^{o}_{i}$ can be approximated by the predicted total-field anomaly
\begin{equation}
\Delta T_{i} = \sum_{j=1}^{M} \, p_{j} a_{ij} \: ,
\label{eq:integral-sum_mag}
\end{equation}
which describes the magnetic induction exerted, at the observation point $(x_{i}, y_{i}, z_{i})$,
by a discrete layer of $M$ dipoles (equivalent sources) defined on the horizontal plane $z = z_{c}$, 
where $p_{j}$ is the magnetic moment intensity (in A~m~$^{2}$)~of the $j$th dipole, 
that has unit volume and is located at the point $(x_{j}, y_{j}, z_{c})$. In equation
\ref{eq:integral-sum_mag}, $a_{ij}$ is the harmonic function
\begin{equation}
a_{ij}
= c_{m} \, \frac{\mu_{0}}{4\pi} \, \hat{\mathbf{F}}^{\top} \mathbf{H}_{ij} \: \hat{\mathbf{u}} \: ,
\label{eq:aij_mag}
\end{equation}
the unit vector
\begin{equation}
\hat{\mathbf{u}} = \begin{bmatrix}
u_x \\
u_y \\
u_z
\end{bmatrix} =
\begin{bmatrix}
\cos(I) \, \cos(D) \\
\cos(I) \, \sin(D) \\
\sin(I)
\end{bmatrix} \: ,
\label{eq:u_hat}
\end{equation}
defines the magnetization direction of all dipoles, with constant inclination $I$ and declination $D$,
$\mu_{0} = 4\pi \, 10^{-7}$ H/m is the magnetic constant, $c_{m} = 10^{9}$ is a factor that transforms
the magnetic induction from Tesla (T) to nanotesla (nT) and $\mathbf{H}_{ij}$ is a $3 \times 3$ matrix 
\begin{equation}
\mathbf{H}_{ij} = \begin{bmatrix}
h^{xx}_{ij} & h^{xy}_{ij} & h^{xz}_{ij} \\
h^{xy}_{ij} & h^{yy}_{ij} & h^{yz}_{ij} \\
h^{xz}_{ij} & h^{yz}_{ij} & h^{zz}_{ij}
\end{bmatrix} \: ,
\label{eq:Hij}
\end{equation}
where 
\begin{equation}
h^{\alpha\beta}_{ij} = 
\begin{cases}
\frac{3 \left( \alpha_{i} - \alpha_{j} \right)^{2}}{r_{ij}^{5}} - \frac{1}{r_{ij}^{3}} \: , \quad \alpha = \beta \\
\frac{3 \left( \alpha_{i} - \alpha_{j} \right) \left( \beta_{i} - \beta_{j} \right)}{r_{ij}^{5}} \: , \quad \alpha \ne \beta
\end{cases} \: , \quad \alpha, \beta = x, y, z \: ,
\label{eq:hij_alpha_beta}
\end{equation}
are the second derivatives of the inverse distance function
\begin{equation}
\frac{1}{r_{ij}} = 
\frac{1}{\sqrt{\left(x_{i} - x_{j} \right)^{2} + 
\left(y_{i} - y_{j} \right)^{2} + \left(z_{i} - z_{c} \right)^{2}}}
\label{eq:1_rij}
\end{equation}
with respect to the coordinates of the observation point $(x_{i}, y_{i}, z_{i})$.

Equation \ref{eq:integral-sum_mag} can be rewritten in matrix notation as follows:
\begin{equation}
\mathbf{d}(\mathbf{p}) = \mathbf{A} \mathbf{p} \: ,
\label{eq:predicted-data-vector_mag}
\end{equation}
where $\mathbf{d}(\mathbf{p})$ is the $N \times 1$ predicted data vector with $i$th element defined
as the predicted total-field anomaly $\Delta T_{i}$ (equation \ref{eq:integral-sum_mag}),
$\mathbf{p}$ is the $M \times 1$ parameter vector whose $j$th element is the magnetic moment intensity
$p_{j}$ of the $j$th dipole and $\mathbf{A}$ is the $N \times M$ sensitivity matrix with element 
$ij$ defined by the harmonic function $a_{ij}$ (equation \ref{eq:aij_mag}).
In the classical equivalent-layer technique, the common approach for 
estimating the parameter vector $\mathbf{p}$ from the observed 
total-field anomaly data $\mathbf{d}^{o}$ is solving the least-squares normal equations
\begin{equation}
\mathbf{A}^{\top}\mathbf{A} \: \mathbf{p} = 
\mathbf{A}^{\top} \mathbf{d}^{o} \: .
\label{eq:normal-equations}
\end{equation}
Equation \ref{eq:normal-equations} is usually solved by first computing the Cholesky 
factor $\mathbf{G}$ of matrix $\mathbf{A}^{\top}\mathbf{A}$ and then using it to solve the linear 
systems \citep[][ p. 262]{golub-vanloan2013}:
\begin{equation}
\begin{split}
\mathbf{G} \mathbf{w} &= \mathbf{A}^{\top}\mathbf{d}^{o} \\
\mathbf{G}^{\top} \tilde{\mathbf{p}} &= \mathbf{w}
\end{split} \: ,
\label{eq:classical-method}
\end{equation}
where $\mathbf{w}$ is a dummy variable.
This approach to estimate the parameter vector will be 
referenced throughout this work as the \textit{classical method}.
The computational cost associated with the classical method can be very high
when dealing with large datasets. In the following subsections, we will show how to 
explore the structure of the sensitivity matrix $\mathbf{A}$ and 
efficiently solve the least-squares normal equations (equation \ref{eq:normal-equations}).


%=====================================================================================================
\subsection{Matrix $\mathbf{A}$ in terms of matrix components $\mathbf{A_{\boldsymbol{\alpha\beta}}}$}
%=====================================================================================================

To access the structure of the sensitivity matrix $\mathbf{A}$ 
(equation \ref{eq:predicted-data-vector_mag}), let us first rewrite its elements 
$a_{ij}$ (equation \ref{eq:aij_mag}) in the following way:
\begin{equation}
\begin{split}
a_{ij} = a^{xx}_{ij} + a^{xy}_{ij} + a^{xz}_{ij} + a^{yy}_{ij} + a^{yz}_{ij} + a^{zz}_{ij} \: ,
\end{split}
\label{eq:aij_mag_expand}
\end{equation}
where
\begin{equation}
a^{\alpha\beta}_{ij} = 
\begin{cases}
c_{m} \, \frac{\mu_{0}}{4\pi} 
\left( F_{\alpha} u_{\beta} \right) h^{\alpha\beta}_{ij} \: &, \quad \alpha = \beta \\
c_{m} \, \frac{\mu_{0}}{4\pi} 
\left( F_{\alpha} u_{\beta} + F_{\beta} u_{\alpha} \right) h^{\alpha\beta}_{ij} \: &, \quad \alpha \ne \beta \\
\end{cases}
\: , \quad \alpha, \beta = x, y, z \: ,
\label{eq:aij_alpha_beta}
\end{equation}
are defined by the elements of $\hat{\mathbf{F}}$ 
(equation \ref{eq:unit_vector_F}), $\hat{\mathbf{u}}$ (equation \ref{eq:u_hat}) and 
$\mathbf{H}_{ij}$ (equations \ref{eq:Hij} and \ref{eq:hij_alpha_beta}).
Then, we can rewrite the sensitivity matrix $\mathbf{A}$ 
(equation \ref{eq:predicted-data-vector_mag}) according to:
\begin{equation}
\mathbf{A} = \mathbf{A_{xx}} + \mathbf{A_{xy}} + \mathbf{A_{xz}} + 
\mathbf{A_{yy}} + \mathbf{A_{yz}} + \mathbf{A_{zz}} \: ,
\label{eq:A_expand}
\end{equation}
where $\mathbf{A_{\boldsymbol{\alpha\beta}}}$ are $N \times M$ matrices with elements 
$ij$ defined by $a^{\alpha\beta}_{ij}$ (equation \ref{eq:aij_alpha_beta}).

Now we can define the structure of $\mathbf{A}$ in terms of its components 
$\mathbf{A_{\boldsymbol{\alpha\beta}}}$ (equation \ref{eq:A_expand}). To do this, 
we consider the particular case in which the observed total-field anomaly is located 
on an $N_{x} \times N_{y}$ 
regular grid of points spaced by $\Delta_{x}$ and $\Delta_{y}$ along the $x$- and $y$-directions,
respectively, on a constant vertical coordinate $z_{0}$. We also consider that the equivalent layer
is formed by one dipole right below each observation point, at the constant coordinate $z_{c}$.
In this case, the number of equivalent sources $M$ is equal to the number of data $N$ and, 
consequently, matrices $\mathbf{A}$ and $\mathbf{A_{\boldsymbol{\alpha\beta}}}$ become 
square ($N \times N$). 
Besides, the horizontal coordinates $x_{i}$ and $y_{i}$ of the observation points 
can be defined by
\begin{equation}
x_{i} = x_{1} + \left[ k(i) - 1 \right] \, \Delta_{x}
\label{eq:xi}
\end{equation}
and
\begin{equation}
y_{i} = y_{1} + \left[ l(i) - 1 \right] \, \Delta_{y} \: ,
\label{eq:yi}
\end{equation}
where $x_{1}$ and $y_{1}$ are the lower limits for $x_{i}$ and $y_{i}$, respectively,
and $k(i)$ and $l(i)$ are integer functions defined according to the orientation
of the data grid (Figure \ref{fig:regular-grids}). 
For $x$-\textit{oriented grids}, the integer functions are given by
\begin{equation}
k(i)  = i - \Bigg\lceil \frac{i}{N_{x}} \Bigg\rceil N_{x} + N_{x}
\label{eq:k-x-oriented}
\end{equation}
and
\begin{equation}
l(i) = \Bigg\lceil \frac{i}{N_{x}} \Bigg\rceil \: .
\label{eq:l-x-oriented}
\end{equation}
For $y$-\textit{oriented grids}, the integer functions are given by
\begin{equation}
k(i) = \Bigg\lceil \frac{i}{N_{y}} \Bigg\rceil
\label{eq:k-y-oriented}
\end{equation}
and
\begin{equation}
l(i) = i - \Bigg\lceil \frac{i}{N_{y}} \Bigg\rceil N_{y} + N_{y} \: .
\label{eq:l-y-oriented}
\end{equation}
In equations \ref{eq:k-x-oriented}--\ref{eq:l-y-oriented}, $\lceil \cdot \rceil$ denotes the ceiling 
function \citep[e.g.,][ p. 67-68]{graham-etal1994}.
Equations \ref{eq:xi}--\ref{eq:l-y-oriented} can also be used to define the coordinates 
$x_{j}$ and $y_{j}$ of the equivalent sources, but with index $j$ instead of $i$.

By using equations \ref{eq:xi}--\ref{eq:l-y-oriented} to define the coordinates $x_{i}$ and 
$y_{i}$ of the observation points and $x_{j}$ and $y_{j}$ of the equivalent sources, we can
rewrite the elements $h^{\alpha\beta}_{ij}$ (equation \ref{eq:hij_alpha_beta}) of matrix 
$\mathbf{H}_{ij}$ (equation \ref{eq:Hij}) as follows:
\begin{equation}
h^{xx}_{ij} = 
\frac{3 \left( \Delta k_{ij} \, \Delta_{x} \right)^{2}}{r_{ij}^{5}} - \frac{1}{r_{ij}^{3}} \: ,
\label{eq:hxx_regular}
\end{equation}
\begin{equation}
h^{yy}_{ij} = 
\frac{3 \left( \Delta l_{ij} \, \Delta_{y} \right)^{2}}{r_{ij}^{5}} - \frac{1}{r_{ij}^{3}} \: ,
\label{eq:hyy_regular}
\end{equation}
\begin{equation}
h^{zz}_{ij} = 
\frac{3 \Delta_{z}^{2}}{r_{ij}^{5}} - \frac{1}{r_{ij}^{3}} \: ,
\label{eq:hzz_regular}
\end{equation}
\begin{equation}
h^{xy}_{ij} = 
\frac{3 \left( \Delta k_{ij} \, \Delta_{x} \right)\left( \Delta l_{ij} \, \Delta_{y} \right)}{r_{ij}^{5}} \: ,
\label{eq:hxy_regular}
\end{equation}
\begin{equation}
h^{xz}_{ij} = 
\frac{3 \left( \Delta k_{ij} \, \Delta_{x} \right) \Delta_{z}}{r_{ij}^{5}}
\label{eq:hxz_regular}
\end{equation}
and
\begin{equation}
h^{yz}_{ij} = 
\frac{3 \left( \Delta l_{ij} \, \Delta_{y} \right) \Delta_{z}}{r_{ij}^{5}} \: ,
\label{eq:hyz_regular}
\end{equation}
where $\Delta_{z} = z_{c} - z_{0}$, 
\begin{equation}
\Delta k_{ij} = \frac{x_{i} - x_{j}}{\Delta_{x}} = k(i) - k(j) \: ,
\label{eq:Delta_kij}
\end{equation}
\begin{equation}
\Delta l_{ij} = \frac{y_{i} - y_{j}}{\Delta_{y}} = l(i) - l(j)
\label{eq:Delta_lij}
\end{equation}
and 
\begin{equation}
\frac{1}{r_{ij}} = 
\frac{1}{\sqrt{\left( \Delta k_{ij} \, \Delta_{x} \right)^{2} + \left( \Delta l_{ij} \, \Delta_{y} \right)^{2} + \Delta_{z}^{2}}} \: .
\label{eq:1_rij_regular}
\end{equation}
Note that the integer functions $k(i)$, $k(j)$, $l(i)$ and $l(j)$ (equations 
\ref{eq:k-x-oriented}--\ref{eq:l-y-oriented}) defining $\Delta k_{ij}$ (equation
\ref{eq:Delta_kij}), $\Delta l_{ij}$ (equation \ref{eq:Delta_lij}) and 
$\tfrac{1}{r_{ij}}$ (equation \ref{eq:1_rij_regular}) assume different 
forms depending on the grid orientation.
Despite of that, it can be shown that
\begin{equation}
\Delta k_{ij} = - \Delta k_{ji} \: ,
\label{eq:Delta_kij_symmetry}
\end{equation}
\begin{equation}
\Delta l_{ij} = - \Delta l_{ji}
\label{eq:Delta_lij_symmetry}
\end{equation}
and 
\begin{equation}
\frac{1}{r_{ij}} = \frac{1}{r_{ji}}
\label{eq:1_rij_symmetry}
\end{equation}
for any grid orientation.

%=================================================================================
\subsection{General structure of matrices $\mathbf{A_{\boldsymbol{\alpha\beta}}}$}
%=================================================================================

By using equations \ref{eq:hxx_regular}--\ref{eq:1_rij_regular} to compute 
$a^{\alpha\beta}_{ij}$ (equation \ref{eq:aij_alpha_beta}), we can show that 
matrices $\mathbf{A_{\boldsymbol{\alpha\beta}}}$ (equation \ref{eq:A_expand}) assume
well-defined structures that can be conveniently
represented with \textit{block indices} $q$ and $p$ \citep{takahashi2020convolutional}.
These indices are defined by the integer functions $\Delta k_{ij}$ and $\Delta l_{ij}$ 
(equations \ref{eq:Delta_kij} and \ref{eq:Delta_lij}), in terms of the indices $i$ 
of the observation points $(x_{i}, y_{i}, z_{0})$ and $j$ of the equivalent sources
$(x_{j}, y_{j}, z_{c})$.
For $x$-\textit{oriented grids} (Figure \ref{fig:regular-grids}), $Q = N_{y}$, $P = N_{x}$ 
and the block indices $q$ and $p$ are given by:
\begin{equation}
q \equiv q(i, j) = \Delta l_{ij}
\label{eq:q-x-oriented}
\end{equation}
and
\begin{equation}
p \equiv p(i, j) = \Delta k_{ij} \: ,
\label{eq:p-x-oriented}
\end{equation}
where $\Delta k_{ij}$ and $\Delta l_{ij}$ (equations \ref{eq:Delta_kij} and \ref{eq:Delta_lij}) 
are defined by integer functions $k(i)$, $k(j)$, $l(i)$ and $l(j)$ given by equations 
\ref{eq:k-x-oriented} and \ref{eq:l-x-oriented}.
For $y$-\textit{oriented grids} (Figure \ref{fig:regular-grids}), $Q = N_{x}$, $P = N_{y}$ and 
the block indices $q$ and $p$ are given by:
\begin{equation}
q \equiv q(i, j) = \Delta k_{ij}
\label{eq:q-y-oriented}
\end{equation}
and
\begin{equation}
p \equiv p(i, j) = \Delta l_{ij} \: ,
\label{eq:p-y-oriented}
\end{equation}
where $\Delta k_{ij}$ and $\Delta l_{ij}$ (equations \ref{eq:Delta_kij} and \ref{eq:Delta_lij}) 
are defined by integer functions $k(i)$, $k(j)$, $l(i)$ and $l(j)$ given by equations 
\ref{eq:k-y-oriented} and \ref{eq:l-y-oriented}.
Equations \ref{eq:q-x-oriented}--\ref{eq:p-y-oriented} show that $q$ varies from $-Q+1$
to $Q-1$ and $p$ from $-P+1$ to $P-1$, regardless of the grid orientation. They differ 
from those presented by \citet{takahashi2020convolutional} due to the absence of the module.

% in review ==========>

Let us consider the small regular grid of $N_{x} = 3$ and $N_{y} = 2$ points shown by
Figure \ref{fig:regular-grids}. This grid may represent observation points 
$(x_{i}, y_{i}, z_{0})$ with constant vertical coordinate $z_{0}$ or equivalent sources
$(x_{j}, y_{j}, z_{c})$ with constant vertical coordinate $z_{c} > z_{0}$. In both cases,
the horizontal coordinates are defined by equations \ref{eq:xi} and \ref{eq:yi}.
Given an index $i$, associated with an observation point, and an index $j$, associated with
an equivalent source, we can compute $\Delta k_{ij}$ (equation \ref{eq:Delta_kij}), 
$\Delta l_{ij}$ (equation \ref{eq:Delta_lij}) and $\tfrac{1}{r_{ij}}$ 
(equation \ref{eq:1_rij_regular}). The matrices $\Delta\mathbf{K}$ and $\Delta\mathbf{L}$ 
having elements $ij$ 
defined by $\Delta k_{ij}$ and $\Delta l_{ij}$, respectively, assume different forms, depending on
the grid orientation. For $x$-oriented grids (Figure \ref{fig:regular-grids}), they are given by:
\begin{equation}
\Delta\mathbf{K} = \begin{bmatrix}
0 &  -1 &  -2 &   0 &  -1 &  -2 \\
1 &   0 &  -1 &   1 &   0 &  -1 \\
2 &   1 &   0 &   2 &   1 &   0 \\
0 &  -1 &  -2 &   0 &  -1 &  -2 \\
1 &   0 &  -1 &   1 &   0 &  -1 \\
2 &   1 &   0 &   2 &   1 &   0 \\
\end{bmatrix}
\label{eq:DK-matrix-x-oriented}
\end{equation}
and
\begin{equation}
\Delta\mathbf{L} = \begin{bmatrix}
0 &   0 &   0 &  -1 &  -1 &  -1 \\
0 &   0 &   0 &  -1 &  -1 &  -1 \\
0 &   0 &   0 &  -1 &  -1 &  -1 \\
1 &   1 &   1 &   0 &   0 &   0 \\
1 &   1 &   1 &   0 &   0 &   0 \\
1 &   1 &   1 &   0 &   0 &   0 \\
\end{bmatrix} \: .
\label{eq:DL-matrix-x-oriented}
\end{equation}
For $y$-oriented grids (Figure \ref{fig:regular-grids}), they are given by:
\begin{equation}
\Delta\mathbf{K} = \begin{bmatrix}
0 &   0 &  -1 &  -1 &  -2 &  -2 \\
0 &   0 &  -1 &  -1 &  -2 &  -2 \\
1 &   1 &   0 &   0 &  -1 &  -1 \\
1 &   1 &   0 &   0 &  -1 &  -1 \\
2 &   2 &   1 &   1 &   0 &   0 \\
2 &   2 &   1 &   1 &   0 &   0 \\
\end{bmatrix}
\label{eq:DK-matrix-y-oriented}
\end{equation}
and
\begin{equation}
\Delta\mathbf{L} = \begin{bmatrix}
0 &  -1 &   0 &  -1 &   0 &  -1 \\
1 &   0 &   1 &   0 &   1 &   0 \\
0 &  -1 &   0 &  -1 &   0 &  -1 \\
1 &   0 &   1 &   0 &   1 &   0 \\
0 &  -1 &   0 &  -1 &   0 &  -1 \\
1 &   0 &   1 &   0 &   1 &   0 \\
\end{bmatrix} \: .
\label{eq:DL-matrix-y-oriented}
\end{equation}
These examples (equations \ref{eq:DK-matrix-x-oriented}--\ref{eq:DL-matrix-y-oriented})
show that different combinations of indices $i$ and $j$ result in integer functions 
$\Delta k_{ij}$ and $\Delta l_{ij}$ (equations \ref{eq:Delta_kij} and \ref{eq:Delta_lij}) 
having the same numerical value. In these cases, not only the numerical values of
the corresponding elements $a^{\alpha\beta}_{ij}$ (equation \ref{eq:aij_alpha_beta}),
but also their associated block indices $q$ and $p$ (equations 
\ref{eq:q-x-oriented}--\ref{eq:p-y-oriented}) are the same. 
The contrary is also true: elements $a^{\alpha\beta}_{ij}$ having different 
associated block indices $q$ and $p$ also have different numerical values.
Because of that, using the alternative notation $a^{\alpha\beta}_{qp}$ to define the elements 
$a^{\alpha\beta}_{ij}$ in terms of its associated block indices $q$ and $p$ is a good
approach to investigate the structure of a given matrix component 
$\mathbf{A_{\boldsymbol{\alpha\beta}}}$ (equation \ref{eq:A_expand}).
This approach allows identifying elements $a^{\alpha\beta}_{ij}$ having the same numerical
value only by inspecting their associated block indices.

Note that, for $x$-oriented grids, matrices $\Delta\mathbf{K}$ (equation \ref{eq:DK-matrix-x-oriented})
and $\Delta\mathbf{L}$ (equation \ref{eq:DL-matrix-x-oriented}) define the block indices
$p$ (equation \ref{eq:p-x-oriented}) and $q$ (equation \ref{eq:q-x-oriented}), respectively.
In this case, they are composed of $Q \times Q$ blocks with $P \times P$ elements each, where 
$Q = N_{y}$ and $P = N_{x}$. 
For $y$-oriented grids, matrices $\Delta\mathbf{K}$ (equation \ref{eq:DK-matrix-y-oriented})
and $\Delta\mathbf{L}$ (equation \ref{eq:DL-matrix-y-oriented}) define the block indices
$q$ (equation \ref{eq:q-y-oriented}) and $p$ (equation \ref{eq:p-y-oriented}), respectively.
In this case, they are also composed of $Q \times Q$ blocks with $P \times P$ elements each, 
but now $Q = N_{x}$ and $P = N_{y}$.
The examples shown by equations \ref{eq:DK-matrix-x-oriented}--\ref{eq:DL-matrix-y-oriented}
also illustrate that, regardless of grid orientation, (i) the block index $q$ is constant 
inside each block; (ii) blocks disposed along the same block diagonal are equal to each other; 
(iii) the block index $p$ is constant on each diagonal of a given block; 
(iv) elements of a given block located on the same diagonal are also equal do each other.
The results obtained with the small grid shown in Figure \ref{fig:regular-grids}
can be easily generalized for larger grids.
Based on the well-defined structure of block indices, we can define 
matrices $\mathbf{A_{\boldsymbol{\alpha\beta}}}$ in a general form
\begin{equation}
\mathbf{A}_{\boldsymbol{\alpha\beta}} = \begin{bmatrix}
\mathbf{A}_{\boldsymbol{\alpha\beta}}^{0}   & \mathbf{A}_{\boldsymbol{\alpha\beta}}^{-1} & \cdots          & \mathbf{A}_{\boldsymbol{\alpha\beta}}^{-Q+1} \\
\mathbf{A}_{\boldsymbol{\alpha\beta}}^{1}   & \ddots          & \ddots          & \vdots           \\ 
\vdots           & \ddots          & \ddots          & \mathbf{A}_{\boldsymbol{\alpha\beta}}^{-1}   \\
\mathbf{A}_{\boldsymbol{\alpha\beta}}^{Q-1} & \cdots          & \mathbf{A}_{\boldsymbol{\alpha\beta}}^{1}  & \mathbf{A}_{\boldsymbol{\alpha\beta}}^{0}
\end{bmatrix}_{N \times N} \: ,
\label{eq:BTTB_A_alpha_beta}
\end{equation}
with blocks $\mathbf{A}_{\boldsymbol{\alpha\beta}}^{q}$, $q = -Q+1, \dots, Q-1$, given by
\begin{equation}
\mathbf{A}_{\boldsymbol{\alpha\beta}}^{q} = \begin{bmatrix}
a^{\alpha\beta}_{q0}   & a^{\alpha\beta}_{q(-1)} & \cdots  & a^{\alpha\beta}_{q(-P+1)} \\
a^{\alpha\beta}_{q1}   & \ddots     & \ddots  & \vdots       \\ 
\vdots      & \ddots     & \ddots  & a^{\alpha\beta}_{q(-1)}   \\
a^{\alpha\beta}_{q(P-1)} & \cdots     & a^{\alpha\beta}_{q1}  & a^{\alpha\beta}_{q0}
\end{bmatrix}_{P \times P} \: ,
\label{eq:Aq_block}
\end{equation}
formed by elements $a^{\alpha\beta}_{qp}$, $p = -P+1, \dots, P-1$.
This well-defined structure (equations \ref{eq:BTTB_A_alpha_beta} and \ref{eq:Aq_block}) 
of matrix components $\mathbf{A_{\boldsymbol{\alpha\beta}}}$ 
(equation \ref{eq:A_expand}) is called Block-Toeplitz Toeplitz-Block (BTTB) 
\citep[e.g., ][ p. 67]{chan-jin2007}.

%=====================================================================================================
\subsection{Detailed structure of matrices $\mathbf{A_{xx}}$, $\mathbf{A_{yy}}$ and $\mathbf{A_{zz}}$}
%=====================================================================================================

Equations \ref{eq:BTTB_A_alpha_beta} and \ref{eq:Aq_block} define the general BTTB
structure of all matrix components $\mathbf{A_{\boldsymbol{\alpha\beta}}}$, but 
there are some differences between them.
Let us consider the matrix component $\mathbf{A}_{\boldsymbol{xx}}$, with elements
$a^{xx}_{ij}$ (equation \ref{eq:aij_alpha_beta}) defined by the second derivative
$h^{xx}_{ij}$ (equation \ref{eq:hxx_regular}). It can be easily verified from equations
\ref{eq:Delta_kij_symmetry} and \ref{eq:1_rij_symmetry} that $h^{xx}_{ij} = h^{xx}_{ji}$.
As a consequence, $a^{xx}_{ij} = a^{xx}_{ji}$, which means that 
\begin{equation}
\mathbf{A}_{\boldsymbol{xx}} = \left( \mathbf{A}_{\boldsymbol{xx}} \right)^{\top}
\label{eq:Axx_symmetry}
\end{equation}
for any grid orientation.
Now, let us investigate the elements $a^{xx}_{qp}$ forming the blocks $\mathbf{A}_{\boldsymbol{xx}}^{q}$.
For $x$-oriented grids (Figure \ref{fig:regular-grids}), the block indices $q$ and $p$ are defined 
by equations \ref{eq:q-x-oriented} and 
\ref{eq:p-x-oriented} and $a^{xx}_{qp}$ can be rewritten as follows:
\begin{equation}
a^{xx}_{qp} = c_{m} \, \frac{\mu_{0}}{4\pi} 
\left( F_{x} u_{x} \right) \frac{3 \left( p \, \Delta_{x} \right)^{2}}{r_{qp}^{5}} - 
\frac{1}{r_{qp}^{3}} \: ,
\label{eq:aqp_xx_x_oriented}
\end{equation}
where
\begin{equation}
\frac{1}{r_{qp}} = 
\frac{1}{\sqrt{\left( p \, \Delta_{x} \right)^{2} + \left( q \, \Delta_{y} \right)^{2} + \Delta_{z}^{2}}} \: .
\label{eq:1_rqp_x_oriented}
\end{equation}
For $y$-oriented grids (Figure \ref{fig:regular-grids}), the block indices $q$ and $p$ are 
defined by equations \ref{eq:q-y-oriented} and 
\ref{eq:p-y-oriented} and $a^{xx}_{qp}$ can be rewritten as follows:
\begin{equation}
a^{xx}_{qp} = c_{m} \, \frac{\mu_{0}}{4\pi} 
\left( F_{x} u_{x} \right) \frac{3 \left( q \, \Delta_{x} \right)^{2}}{r_{qp}^{5}} - 
\frac{1}{r_{qp}^{3}} \: ,
\label{eq:aqp_xx_y_oriented}
\end{equation}
where
\begin{equation}
\frac{1}{r_{qp}} = 
\frac{1}{\sqrt{\left( q \, \Delta_{x} \right)^{2} + \left( p \, \Delta_{y} \right)^{2} + \Delta_{z}^{2}}} \: .
\label{eq:1_rqp_y_oriented}
\end{equation}
From equations \ref{eq:aqp_xx_x_oriented}--\ref{eq:1_rqp_y_oriented}, we can easily verify that
\begin{equation}
\mathbf{A}_{\boldsymbol{xx}}^{q} = \mathbf{A}_{\boldsymbol{xx}}^{(-q)}
\label{eq:Axx_q_external_block_symmetry}
\end{equation}
and
\begin{equation}
\mathbf{A}_{\boldsymbol{xx}}^{q} = \left(\mathbf{A}_{\boldsymbol{xx}}^{q} \right)^{\top} \: .
\label{eq:Axx_q_internal_block_symmetry}
\end{equation}
Note that these symmetries are valid for 
any grid orientation.
From this results we conclude the matrix component 
$\mathbf{A}_{\boldsymbol{xx}}$ is \textit{symmetric-Block-Toeplitz symmetric-Toeplitz-Block} 
for any grid orientation.
The same reasoning can be used to show that matrices $\mathbf{A}_{\boldsymbol{yy}}$ and
$\mathbf{A}_{\boldsymbol{zz}}$ also have this symmetric structure.

%==========================================================
\subsection{Detailed structure of matrix $\mathbf{A_{xy}}$}
%==========================================================

Let $\mathbf{A}_{\boldsymbol{xy}}$ be a matrix component with elements
$a^{xy}_{ij}$ (equation \ref{eq:aij_alpha_beta}) defined by the second derivative
$h^{xy}_{ij}$ (equation \ref{eq:hxy_regular}). It can be easily verified from equations
\ref{eq:Delta_kij_symmetry}--\ref{eq:1_rij_symmetry} that $h^{xy}_{ij} = h^{xy}_{ji}$.
As a consequence, $a^{xy}_{ij} = a^{xy}_{ji}$, which means that 
\begin{equation}
\mathbf{A}_{\boldsymbol{xy}} = \left( \mathbf{A}_{\boldsymbol{xy}} \right)^{\top}
\label{eq:Axy_symmetry}
\end{equation}
for any grid orientation.
For $x$-oriented grids (Figure \ref{fig:regular-grids}), the block indices $q$ and $p$ are defined 
by equations \ref{eq:q-x-oriented} and 
\ref{eq:p-x-oriented} and $a^{xy}_{qp}$ can be rewritten as follows:
\begin{equation}
a^{xy}_{qp} = c_{m} \, \frac{\mu_{0}}{4\pi} 
\left( F_{x} u_{y} + F_{y} u_{x} \right) \frac{3 \left( p \, \Delta_{x} \right)\left( q \, \Delta_{y} \right)}{r_{qp}^{5}}
\: ,
\label{eq:aqp_xy_x_oriented}
\end{equation}
with $\tfrac{1}{r_{qp}}$ defined by equation \ref{eq:1_rqp_x_oriented}.
For $y$-oriented grids (Figure \ref{fig:regular-grids}), the block indices $q$ and $p$ are 
defined by equations \ref{eq:q-y-oriented} and 
\ref{eq:p-y-oriented} and $a^{xy}_{qp}$ can be rewritten as follows:
\begin{equation}
a^{xy}_{qp} = c_{m} \, \frac{\mu_{0}}{4\pi} 
\left( F_{x} u_{y} + F_{y} u_{x} \right) \frac{3 \left( q \, \Delta_{x} \right)\left( p \, \Delta_{y} \right)}{r_{qp}^{5}} \: ,
\label{eq:aqp_xy_y_oriented}
\end{equation}
with $\tfrac{1}{r_{qp}}$ defined by equation \ref{eq:1_rqp_y_oriented}.
From equations \ref{eq:1_rqp_x_oriented}, \ref{eq:1_rqp_y_oriented}, \ref{eq:aqp_xy_x_oriented} 
and \ref{eq:aqp_xy_y_oriented}, we can show that
\begin{equation}
\mathbf{A}_{\boldsymbol{xy}}^{q} = -\mathbf{A}_{\boldsymbol{xy}}^{(-q)}
\label{eq:Axy_q_external_block_symmetry}
\end{equation}
and 
\begin{equation}
\mathbf{A}_{\boldsymbol{xy}}^{q} = -\left( \mathbf{A}_{\boldsymbol{xy}}^{q} \right)^{\top} \: .
\label{eq:Axy_q_internal_block_symmetry}
\end{equation}
Note that these symmetries are valid for any grid orientation.
From this results we conclude the matrix component 
$\mathbf{A}_{\boldsymbol{xy}}$ is \textit{skew symmetric-Block-Toeplitz skew symmetric-Toeplitz-Block} 
for any grid orientation.

%==================================================================================
\subsection{Detailed structure of matrices $\mathbf{A_{xz}}$ and $\mathbf{A_{yz}}$}
%==================================================================================

Let $\mathbf{A}_{\boldsymbol{xz}}$ be a matrix component with elements
$a^{xz}_{ij}$ (equation \ref{eq:aij_alpha_beta}) defined by the second derivative
$h^{xz}_{ij}$ (equation \ref{eq:hxz_regular}). It can be easily verified from equations
\ref{eq:Delta_kij_symmetry}--\ref{eq:1_rij_symmetry} that $h^{xz}_{ij} = -h^{xz}_{ji}$.
As a consequence, $a^{xz}_{ij} = -a^{xz}_{ji}$, which means that 
\begin{equation}
\mathbf{A}_{\boldsymbol{xz}} = -\left( \mathbf{A}_{\boldsymbol{xz}} \right)^{\top}
\label{eq:Axz_symmetry}
\end{equation} 
for any grid orientation.
For $x$-oriented grids (Figure \ref{fig:regular-grids}), the block indices $q$ and $p$ are defined 
by equations \ref{eq:q-x-oriented} and 
\ref{eq:p-x-oriented} and $a^{xz}_{qp}$ can be rewritten as follows:
\begin{equation}
a^{xz}_{qp} = c_{m} \, \frac{\mu_{0}}{4\pi} 
\left( F_{x} u_{z} + F_{z} u_{x} \right) \frac{3 \left( p \, \Delta_{x} \right) \Delta_{z}}{r_{qp}^{5}}
\: ,
\label{eq:aqp_xz_x_oriented}
\end{equation}
with $\tfrac{1}{r_{qp}}$ defined by equation \ref{eq:1_rqp_x_oriented}.
In this case, we can see that
\begin{equation}
\mathbf{A}_{\boldsymbol{xz}}^{q} = \mathbf{A}_{\boldsymbol{xz}}^{(-q)}
\label{eq:Axz_q_external_block_symmetry_x_oriented}
\end{equation}
and 
\begin{equation}
\mathbf{A}_{\boldsymbol{xz}}^{q} = -\left( \mathbf{A}_{\boldsymbol{xz}}^{q} \right)^{\top} \: .
\label{eq:Axz_q_internal_block_symmetry_x_oriented}
\end{equation}
This structure is called \textit{symmetric-Block-Toeplitz skew symmetric-Toeplitz-Block} and is 
valid only for $x$-oriented grids.
For $y$-oriented grids (Figure \ref{fig:regular-grids}), the block indices $q$ and $p$ are 
defined by equations \ref{eq:q-y-oriented} and 
\ref{eq:p-y-oriented} and $a^{xz}_{qp}$ can be rewritten as follows:
\begin{equation}
a^{xz}_{qp} = c_{m} \, \frac{\mu_{0}}{4\pi} 
\left( F_{x} u_{z} + F_{z} u_{x} \right) \frac{3 \left( q \, \Delta_{x} \right) \Delta_{z}}{r_{qp}^{5}} \: ,
\label{eq:aqp_xz_y_oriented}
\end{equation}
with $\tfrac{1}{r_{qp}}$ defined by equation \ref{eq:1_rqp_y_oriented}.
Now, we conclude that
\begin{equation}
\mathbf{A}_{\boldsymbol{xz}}^{q} = -\mathbf{A}_{\boldsymbol{xz}}^{(-q)}
\label{eq:Axz_q_external_block_symmetry_y_oriented}
\end{equation}
and 
\begin{equation}
\mathbf{A}_{\boldsymbol{xz}}^{q} = \left( \mathbf{A}_{\boldsymbol{xz}}^{q} \right)^{\top} \: .
\label{eq:Axz_q_internal_block_symmetry_y_oriented}
\end{equation}
This structure is called \textit{skew symmetric-Block-Toeplitz symmetric-Toeplitz-Block} and is 
valid only for $y$-oriented grids.

The same reasoning can be followed to show that
\begin{equation}
\mathbf{A}_{\boldsymbol{yz}} = -\left( \mathbf{A}_{\boldsymbol{yz}} \right)^{\top}
\label{eq:Ayz_symmetry}
\end{equation} 
for any grid orientation. Besides, we can also show that
\begin{equation}
\mathbf{A}_{\boldsymbol{yz}}^{q} = -\mathbf{A}_{\boldsymbol{yz}}^{(-q)}
\label{eq:Ayz_q_external_block_symmetry_x_oriented}
\end{equation}
and 
\begin{equation}
\mathbf{A}_{\boldsymbol{yz}}^{q} = \left( \mathbf{A}_{\boldsymbol{yz}}^{q} \right)^{\top}
\label{eq:Ayz_q_internal_block_symmetry_x_oriented}
\end{equation}
for $x$-oriented grids (\textit{skew symmetric-Block-Toeplitz symmetric-Toeplitz-Block}), while
\begin{equation}
\mathbf{A}_{\boldsymbol{yz}}^{q} = \mathbf{A}_{\boldsymbol{yz}}^{(-q)}
\label{eq:Ayz_q_external_block_symmetry_y_oriented}
\end{equation}
and 
\begin{equation}
\mathbf{A}_{\boldsymbol{yz}}^{q} = -\left( \mathbf{A}_{\boldsymbol{yz}}^{q} \right)^{\top}
\label{eq:Ayz_q_internal_block_symmetry_y_oriented}
\end{equation}
for $y$-oriented grids (\textit{symmetric-Block-Toeplitz skew symmetric-Toeplitz-Block}).

%======================================================================================
%\subsection{Standard Conjugate Gradient Least Squares (CGLS) method}
\subsection{Convolutional equivalent layer}
%======================================================================================

The computational cost associated with the classical method to estimate the parameter 
vector $\mathbf{p}$ by solving the linear system \ref{eq:normal-equations} can be very high 
or even prohibitive when dealing with large data sets. In these cases, a well-known alternative
is solving the normal equations (equation \ref{eq:normal-equations}) iteratively by 
using the \textit{standard Conjugate Gradient Least Squares (CGLS) method}:

\begin{algorithm}[H]
	Input: $\mathbf{A}$ and $\mathbf{d}^{o}$.
	
	Output: Estimated parameter vector $\tilde{\mathbf{p}}$.
	
	Set $it = 0$, $\tilde{\mathbf{p}}_{(it)} = \mathbf{0}$, $\mathbf{c}_{(it-1)} = \mathbf{0}$, $\beta_{(it)} = 0$, $\mathbf{s}_{(it)} = \mathbf{d}^{o}$ and $\mathbf{r}_{(it)} = \mathbf{A}^{\top} \mathbf{s}_{(it)}$.
	
	1 - If $it > 0$, $\beta_{(it)} = \dfrac{\| \mathbf{r}_{(it)} \|_{2}^{2}}{\| \mathbf{r}_{(it - 1)} \|_{2}^{2}}$
	
	2 - $\mathbf{c}_{(it)} = \mathbf{r}_{(it)} + \beta_{(it)} \, \mathbf{c}_{(it - 1)}$
	
	3 - $\alpha_{(it)} = \dfrac{{\| \mathbf{r}_{(it)}\|_{2}^{2}}}{\| \mathbf{A} \, \mathbf{c}_{(it)} \|_{2}^{2}}$
	
	4 - $\tilde{\mathbf{p}}_{(it + 1)} = \tilde{\mathbf{p}}_{(it)} + \alpha_{(it)} \, \mathbf{c}_{(it)}$
	
	5 - $\mathbf{s}_{(it + 1)} = \mathbf{s}_{(it)} - \alpha_{(it)} \, \mathbf{A} \, \mathbf{c}_{(it)}$
	
	6 - $\mathbf{r}_{(it + 1)} = \mathbf{A}^{\top} \, \mathbf{s}_{(it + 1)}$
	
	7 - $it = it + 1$
	
	8 - Repeat previous steps until convergence.
	
	\caption{Standard CGLS pseudocode \citep[][ p. 166]{aster2019parameter}.}
\label{al:std-cgls-algorithm}
\end{algorithm}

Setting a convergence criteria based on the minimum tolerance of the residuals is a good 
option to carry out this algorithm efficiently and still obtaining very good results. 
Another possibility is to set an invariance to the Euclidean norm of residuals between 
iterations, which would increase algorithm runtime, but with smaller residuals. 
We chose the latter option, as we could achieve better results.

Note that the standard CGLS solution (Algorithm \ref{al:std-cgls-algorithm}) requires 
neither inverse matrix nor matrix-matrix product. Instead, it only requires: one matrix-vector 
product out of the loop and two matrix-vector products per iteration (in steps 3 and 6). 
These products can be efficiently computed by using the 2D FFT, as a discrete convolution
(see Appendix A). \citet{takahashi2020convolutional} used this approach
to develop an efficient algorithm for gravity data processing. This modified approach in which
the standard CGLS method is modified to efficiently compute the matrix-vector products will be 
referenced throughout this work as the \textit{convolutional equivalent layer method}.

%======================================================================================
\subsection{Computational performance}
%======================================================================================

In this sections we compare the efficiency of the classical (equation \ref{eq:classical-method}), 
standard CGLS (Algorithm \ref{al:std-cgls-algorithm}) and the convolutional equivalent 
layer method (Algorithm \ref{al:std-cgls-algorithm} with matrix-vector products computed 
according to Appendix A). To do this, we compute the total number of 
\textit{flops} associated to them \citep[][ p. 12]{golub-vanloan2013}.

For the classical method, we have $\tfrac{1}{2} N^3$ flops to compute the lower triangle of
$\mathbf{A}^{\top}\mathbf{A}$; $\tfrac{1}{3} N^3$ flops to compute the Cholesky factor
$\mathbf{G}$ of $\mathbf{A}^{\top}\mathbf{A}$ \citep[][ p.~164]{golub-vanloan2013};
$2 \, N^2$ flops to compute the matrix-vector product $\mathbf{A}^{\top} \mathbf{d}^{o}$;
and $2 \, N^2$ flops to solve the triangular systems given by equation \ref{eq:classical-method}
\citep[][ p.~106]{golub-vanloan2013}. The resultant flop count for the classical method is
\begin{equation}
f_{classical} =  \dfrac{5}{6} N^{3} + 4 \, N^{2}\: .
\label{eq:flops-classical-method}
\end{equation}

%======================================================================================
%\subsection{CGLS flops count}
%======================================================================================

For the standard CGLS method (Algorithm \ref{al:std-cgls-algorithm}) we have $2 \, N^2$ to compute
the matrix-vector product $\mathbf{A}^{\top} \mathbf{s}_{(it)}$ out of the loop;
$4 \, N$ in step 1; $2 \, N$ in step 2; $2 \, N^2 + 2 \, N$ in step 3; $2 \, N$ in step 4;
$2 \, N$ in step 5; and $2 \, N^2$ in step 6. The resultant flop count is given by:
\begin{equation}
f_{cgls} =  2 N^{2} + it \, (4 N^{2} + 12 N) \: .
\label{eq:flops-standard-cgls}
\end{equation}

%======================================================================================
%\subsection{Our modified CGLS flops count}
%======================================================================================

To compute the flops count of our method, we need only to replace the flops associated with 
matrix-vector products in the standard CGLS method by those associated with
2D convolution defined in Appendix A, which consists of $\kappa  \, 4 N \log_2(4N)$ flops to
compute the 2D FFT for each matrix $\mathbf{L}_{\boldsymbol{\alpha\beta}}$ (equation 
\ref{eq:L_alpha_beta}); $\kappa  \, 4 N \log_2(4N)$ flops to compute 
$\mathbf{F}_{2Q} \, \mathbf{V} \, \mathbf{F}_{2P}$ via 2D FFT; $24 \, N$ flops to compute the 
Hadamard product $\mathbf{L} \circ \left(\mathbf{F}_{2Q} \, \mathbf{V} \, \mathbf{F}_{2P} \right)$; 
and $\kappa  \, 4 N \log_2(4N)$ flops to compute the IDFT in equation 
\ref{eq:2d-discrete-convolution-complete}. We use $\kappa = 5$ for the \emph{radix-2} algorithm
\citep[][ p.~15]{vanloan1992}. By replacing these flops into Algorithm \ref{al:std-cgls-algorithm},
we obtain the complete number of flops
\begin{equation}
f_{conv} =  \kappa  \, 16 N \log_2(4 N) + 24 N + it \, (\kappa  \, 16 N \log_2 (4 N) + 60 N) \: .
\label{eq:flops-convolutional-method}
\end{equation}

Figure \ref{fig:flops} shows a comparison between 
$f_{classical}$ (equation \ref{eq:flops-classical-method}), 
$f_{cgls}$ (equation \ref{eq:flops-standard-cgls}) and 
$f_{conv}$ (equation \ref{eq:flops-convolutional-method})
for different numbers of observation points up to $1,000,000$. As we can see, 
the total flops count associated with our method is $10^7$ orders of magnitude smaller 
than that associated with the classical method and $10^3$ orders of magnitude smaller than
that associated with the standard CGLS method by using a maximum number
of iterations $N^{it} = 50$. 

Figure \ref{fig:solve_time} shows the time necessary to build matrix $\mathbf{A}$ 
(equation \ref{eq:A_expand}) and solve the linear system for $N$ varying up to $10,000$. 
With $N = 10,000$, the classical method takes more than sixty-three seconds, the standard 
CGLS more than twelve seconds, while our method takes only half a second. 
The CPU used for this test was a Intel Core i7-7700HQ@2.8GHz.

%In Figure \ref{fig:sources_time} a comparison between the time to complete the task to calculate the first column of the BCCB matrix embbeded from the from $\mathbf{A}$ (equation \ref{eq:aij_mag}) by using only one equivalent source, i.e., calculating all six first column of the second derivatives matrices from $\mathbf{H}$ (equation \ref{eq:Hi}) and using four equivalent sources to calculate the four necessary columns from the non-symmetric matrix $\mathbf{A}$ (equation \ref{eq:aij_mag}). Although, very similar in time, with one source a small advantage can be observed as the number of data $N$ increases and goes beyond $N = 200,000$. This test was done from $N = 10,000$ to $N = 700,000$ with increases of $5,625$ observation points.

Table \ref{tab:RAM-usage} shows a comparison between the RAM memory storage 
associated with each method. The classical and standard CGLS methods have to store the whole 
matrix $\mathbf{A}$ (equation \ref{eq:A_expand}). For example, a dataset with 
$N = 10,000$ observation points has an associated sensitivity matrix $\mathbf{A}$ formed by 
$N^2 = 100,000,000$ elements and takes approximately $763$ Megabytes of memory (8 bytes per element). 
Using the same number of observation points $N = 10,000$, our method requires only 
$1.831$ Megabytes to store the first columns of the BCCB matrices
$\mathbf{C}_{\boldsymbol{\alpha\beta}}$ (equation \ref{eq:w_alpha_beta}) and 
$0.6104$ Megabytes to store the complex matrix $\mathbf{L}$ (equation \ref{eq:L}) 
(16 bytes per element). For a bigger dataset with $N = 1,000,000$, the amount of necessary RAM 
goes to $7,629,395$, $183.096$ and $61.035$ Megabytes, respectively.

\section{Application to synthetic data}

Our convolutional equivalent layer method requires a regular data grid located on a 
horizontal and flat observation surface.
Here, we evaluate the performance of our method by applying it to simulated airborne magnetic 
surveys formed by
i) a regular data grid on a flat surface;
ii) irregular data grids on a flat surface; and 
iii) regular data grid on undulating surfaces.
Note that the simulated surveys in (ii) and (iii) violate the premises of our method. 

%=======================================
\subsection*{Simulated airborne surveys}
%=======================================

The upper and middle rows in Figure \ref{fig:synthetic_data_comparison_v2} show, respectively, 
the simulated flight patterns and noise-corrupted total-field anomalies of the airborne magnetic 
surveys used in our tests. The lower row in Figure \ref{fig:synthetic_data_comparison_v2} shows 
the true upward-continued total-field anomalies at $z = -1, \, 300$ m.
All magnetic data (middle and lower rows in Figure \ref{fig:synthetic_data_comparison_v2}) 
are produced by the same three synthetic bodies: two prisms and one sphere with 
constant total-magnetization vector having inclination, declination and intensity of 
$0^{\circ}$, $45^{\circ}$, and $2.8284$ A/m, respectively. 
The simulated main geomagnetic field has inclination and declination of $10^{\circ}$ and $37^{\circ}$,
respectively. 

%% regular grid on a flat surface

Figure \ref{fig:synthetic_data_comparison_v2}a shows the simulated airborne survey on
a regular grid of $100 \times 50$ observation points (totaling  $N = 5,\, 000$ observation points),
with a grid spacing of $\Delta x = 101.01$ m and $\Delta y = 163.265$ m along the
$x$- and $y$-axis, respectively.
The noise-corrupted total-field anomaly (middle panel) is calculated at $z = -900$ m, with a
pseudorandom Gaussian noise having null mean and standard deviation of $0.2961$ nT.

%% irregular grids on a flat surface

Figures \ref{fig:synthetic_data_comparison_v2}b and \ref{fig:synthetic_data_comparison_v2}c 
show the simulated surveys on irregular grids obtained by perturbing the horizontal coordinates
of the regular grid (upper panel in Figure \ref{fig:synthetic_data_comparison_v2}a).
For the survey shown in Figure \ref{fig:synthetic_data_comparison_v2}b, the $x$ and $y$ coordinates 
are perturbed with sequences of pseudorandom Gaussian noises having null mean and standard deviations
equal to $20\%$ of the corresponding grid spacing, which results in
absolute values of $20.2$ m and $32.6$ m, along the $x$- and $y$-directions, respectively.
For the survey shown in Figure \ref{fig:synthetic_data_comparison_v2}c, the standard deviations
are equal to $30\%$ of the corresponding grid spacing, which results in absolute values of 
$30.3$ m and $49.0$ m along the $x$- and $y$-directions, respectively.
Their noise-corrupted total-field anomalies (middle panels in Figures 
\ref{fig:synthetic_data_comparison_v2}b and \ref{fig:synthetic_data_comparison_v2}c) are calculated 
on their corresponding irregular grids, on a flat observation surface at $z = -900$ m, 
with a pseudorandom Gaussian noise having null mean and standard deviation of $0.2961$ nT.

%% regular grid on undulating surfaces

Figures \ref{fig:synthetic_data_comparison_v2}d and \ref{fig:synthetic_data_comparison_v2}e 
show the simulated surveys on the same regular grid shown in Figure 
\ref{fig:synthetic_data_comparison_v2}a (upper panel). The difference is that 
they are no longer on a flat, but on undulating surfaces.
For the survey shown in Figure \ref{fig:synthetic_data_comparison_v2}d, the $z$ coordinates 
of the undulating surface are defined by a sequence of pseudorandom Gaussian noise having mean 
$-900$ m and standard deviation equal to $5\%$ of $900$ m, which corresponds to $45$ m.
For the survey shown in Figure \ref{fig:synthetic_data_comparison_v2}e, the standard deviation 
is equal to $10\%$ of $900$ m, which corresponds to $90$ m.
The noise-corrupted total-field anomalies of these simulated surveys (middle panels in Figures 
\ref{fig:synthetic_data_comparison_v2}d and \ref{fig:synthetic_data_comparison_v2}e) are calculated 
on their corresponding undulating surfaces (upper panels in Figures 
\ref{fig:synthetic_data_comparison_v2}d and \ref{fig:synthetic_data_comparison_v2}e),
on the same regular grid shown in Figure \ref{fig:synthetic_data_comparison_v2}a,
with a pseudorandom Gaussian noise having null mean and standard deviation of $0.2961$ nT.


%% Regular grid
%======================================================================================
\subsection*{Tests with a regular data grid on a flat surface}
%======================================================================================

Figure \ref{fig:synthetic_residuals_convergence_comparison_v2} show the 
difference between the simulated (middle row in Figure \ref{fig:synthetic_data_comparison_v2})
and predicted data (not shown) obtained by using the classical (the upper row) and 
our method (the middle row). From now on, we designate this difference as data residuals. 
The lower row in Figure \ref{fig:synthetic_residuals_convergence_comparison_v2} shows the 
convergence curve of our method.

The data residuals using the classical method (equation \ref{eq:classical-method})  
are shown in the upper panel of Figure \ref{fig:synthetic_residuals_convergence_comparison_v2}a, 
with mean $0.3627$ nT and standard deviation $0.2724$ nT. This process took $17.10$ seconds.
Using our method, the data residuals (the middle panel in Figure
\ref{fig:synthetic_residuals_convergence_comparison_v2}a) have mean $0.5223$ nT and standard
deviation $0.4323$ nT. In this case, however, the processing time was only $0.18$ seconds.
As expected, the Euclidean norm of the data residuals produced by our method 
(lower panel in Figure \ref{fig:synthetic_residuals_convergence_comparison_v2}a) decreases 
and the convergence criterion is satisfied close to iteration $50$. 
This result shows that, in practice, our method converges way before $N$ iterations,
where $N$ is the number of observations.
Setting the convergence to $N$ iterations, besides being unnecessary, it also demands a larger 
computer processing time, even in this synthetic test with a small number 
of $N = 5,\, 000$ observations.

%======================================================================================
\subsection*{Tests with irregular data grids on a flat surface}
%======================================================================================

Figure \ref{fig:synthetic_residuals_convergence_comparison_v2}b shows the results obtained
with the irregular data grid perturbed by using $20\%$ of the regular grid spacing.
In this Figure we can see that the data residuals 
using the classical method (upper panel) yield a good data fit with mean $0.3630$ nT and standard
deviation $0.2731$ nT. 
Using our method, the data residuals (middle panel in Figure 
\ref{fig:synthetic_residuals_convergence_comparison_v2}b) also produced an acceptable data 
fitting with mean of  $0.7147$ nT and standard deviation of $0.5622$ nT. 
The Euclidean norm of the data residuals obtained by our method 
(lower panel in Figure \ref{fig:synthetic_residuals_convergence_comparison_v2}b) decreases, 
as expected, and converges to a constant value close to iteration $50$. 

Figure \ref{fig:synthetic_residuals_convergence_comparison_v2}b shows the results obtained
with the irregular data grid perturbed by using $30\%$ of the regular grid spacing.
This figure shows that the data residuals 
obtained by the classical method (upper panel) produced an acceptable data fitting, having mean 
$0.3634$ nT and standard deviation $0.2735$ nT. 
%Using our method, the data residuals (middle panel in Figure 
%\ref{fig:synthetic_residuals_convergence_comparison_v2}c) with mean $0.9788$ nT and 
%standard deviation $0.7462$ nT produced a worse data fitting.
%The convergence of our method (lower panel in Figure 
%\ref{fig:synthetic_residuals_convergence_comparison_v2}c) shows that, 
%similarly to the previous results, the Euclidean norm of the residuals decreases; however it starts
%increasing without achieving an invariance. Hence, the convergence is not achieved. 
%
%\textbf{Alternative text} $\rightarrow$
%
Using our method, the data residuals (middle panel in Figure 
\ref{fig:synthetic_residuals_convergence_comparison_v2}c) with mean $0.9186$ nT and 
standard deviation $0.7354$ nT also produced a good data fitting.
The convergence of our method (lower panel in Figure 
\ref{fig:synthetic_residuals_convergence_comparison_v2}c) shows that, 
similarly to the previous results, the Euclidean norm of the residuals decreases; however it starts
increasing without achieving an invariance. In this case, we have to stop the algorithm
at iteration $40$. Note that this good result was obtained by using a very perturbed
data grid (upper panel in Figure \ref{fig:synthetic_data_comparison_v2}c).
%
%$\leftarrow$ \textbf{Alternative text}

%======================================================================================
\subsection*{Tests with regular data grid an undulating surfaces}
%======================================================================================

Figure \ref{fig:synthetic_residuals_convergence_comparison_v2}d shows the results obtained
with data on the undulating surface varying $5\%$ of $z = 900$ m.
In this case, the data residuals either using the classical method 
(upper panel in Figure \ref{fig:synthetic_residuals_convergence_comparison_v2}d) or
our method (middle panel in Figure \ref{fig:synthetic_residuals_convergence_comparison_v2}d) reveal
acceptable data fittings.
Using the classical method, data residuals have mean $0.3712$ nT and standard deviation $0.2870$ nT.
Using our method, they have mean $0.9542$ nT and standard deviation $0.8943$ nT. 
Likewise, the Euclidean norm of the data residuals produced by our method 
(lower panel in Figure \ref{fig:synthetic_residuals_convergence_comparison_v2}d) decreases up to 
iteration 50 and reaches an invariance in the subsequent iterations.

Figure \ref{fig:synthetic_residuals_convergence_comparison_v2}e shows the results obtained
with data on the undulating surface varying $10\%$ of $z = 900$ m.
By using the classical approach, the data residuals (upper panel in 
Figure \ref{fig:synthetic_residuals_convergence_comparison_v2}e) 
yielded a good data fitting, with mean $0.3865$ nT and standard deviation $0.3216$ nT. 
By using our method, the data residuals (middle panel in 
Figure \ref{fig:synthetic_residuals_convergence_comparison_v2}e) yielded a worse data fitting 
with mean $1.6109$ nT and standard deviation $1.6231$ nT.
The convergence curve (lower panel in Figure \ref{fig:synthetic_residuals_convergence_comparison_v2}e)
reveals the inadequacy of our method in dealing with observations on rugged surfaces, as 
the Euclidean norm of the data residuals decreases slower than in previous tests. 
We stress that, in this test, the undulating surface (upper panel in Figure 
\ref{fig:synthetic_data_comparison_v2}e) varies in a broad range from $z = - 570$ m to about 
$z = -1,\, 230$ m. Thus, this simulated airborne magnetic survey greatly violates the requirement 
of a flat observation surface demanded by our method.

Although our method is formulated to deal with magnetic observations measured on 
a horizontally regular grid, on a flat surface, the results obtained with synthetic 
data show that our method is robust in dealing either with irregular grids in the 
horizontal directions or with uneven surfaces.
However, the robustness of our method has limitations.
The performance limitation of our method depends on the degree of 
departure of the (i) $x$- and $y$-coordinates from those of the closest 
regular grid and (ii) the $z$ coordinates from a constant value.
High departures in the $x$-, $y$, and $z$-coordinates lead to unacceptable 
data fittings (large data residuals), as shown the middle panels in Figures 
\ref{fig:synthetic_residuals_convergence_comparison_v2}c and
\ref{fig:synthetic_residuals_convergence_comparison_v2}e.

%======================================================================================
\subsection*{Magnetic data processing}
%======================================================================================

We performed the upward continuations of the synthetic total-field anomalies 
(middle row in Figure \ref{fig:synthetic_data_comparison_v2}) by using 
the classical method, our convolutional equivalent layer method, and 
the classical approach in the Fourier domain,
which consists in computing the Fourier transform of the total-field anomaly 
\citep[e.g.,][ p. 317]{blakely1996}. 

Figure \ref{fig:synthetic_upward_residuals_comparison_v2} shows the differences
between the true upward-continued total-field anomalies (lower row in Figure
\ref{fig:synthetic_data_comparison_v2}) and the predicted upward-continued total-field 
anomalies (not shown). We conveniently denote these differences as continuation
residuals.

%% Classical and our method 

Figure \ref{fig:synthetic_upward_residuals_comparison_v2} shows that the continuation residuals 
obtained by using the classical method (upper row) and our method (middle row) are 
similar to each other in most of the tests.
One exception is the synthetic test with data over an undulating surface 
(Figures \ref{fig:synthetic_data_comparison_v2}e and 
\ref{fig:synthetic_residuals_convergence_comparison_v2}e), which greatly violates the 
requirement of a flat observation surface demanded by our method.
Note that the maximum absolute value of the continuation residuals produced by using our 
method (middle panel in Figure \ref{fig:synthetic_upward_residuals_comparison_v2}e) 
are $\approx 2.5$ times greater than those produced by the classical method 
(upper panel in Figure \ref{fig:synthetic_upward_residuals_comparison_v2}e).
Besides, they are generally concentrated at the boundaries of the study area.

%%  Fourier 

In contrast, the continuation residuals obtained by using the 
classical Fourier approach (lower row in Figure \ref{fig:synthetic_upward_residuals_comparison_v2})
are, in most of the tests, approximately $6$ times greater than those produced by the classical method 
(upper row in Figure \ref{fig:synthetic_upward_residuals_comparison_v2}) and $4$ times greater than
those produced by our method (middle row in Figure \ref{fig:synthetic_upward_residuals_comparison_v2}).
Note that, similarly to our method, the maximum absolute values of the continuation residuals 
obtained by using the classical Fourier approach are located at the boundaries of the simulated area.
However, the values are significantly higher.

We call attention to the following aspects:
In applying the classical method, our method, or the classical Fourier approach, we do not expand 
the data by using a padding scheme.
The data residuals (upper and middle rows in Figure 
\ref{fig:synthetic_residuals_convergence_comparison_v2}) 
and the continuation residuals (Figure \ref{fig:synthetic_upward_residuals_comparison_v2}) are
shown without removing the edge effects. 
The computational time required by our method is much lower than that required by the classical method
and has the same order of magnitude of that required by the classical Fourier approach.
However, the classical Fourier approach shows upward-continued data with strong border effects 
if no padding scheme is applied to expand the data. 



\section{Application to field data}

We applied the convolutional equivalent layer method to the aeromagnetic data of Carajás, 
northern Brazil.
The survey is composed of $131$ flight lines along north-south direction with line spacing of 
$\Delta y = 3,000$ m. 
Data were measured with average spacing $\Delta x = 7.65$ m along lines,
with an average distance to the ground of $900$ m. 
The total number of observation points is $N = 6,081,345$. Figure \ref{fig:carajas_residuals_comparison}a
shows the observed total-field anomaly data over the study area.

We compare the results obtained with an interpolated regular grid of $10,000 \times 131$ points, 
by using the nearest neighbor algorithm, and a decimated irregular grid, also with $10,000 \times 131$
points. In both cases the total $N = 1,310,000$ observation points are in the original undulating surface of the flight lines. 
The decimated grid was generated by 
choosing the nearest observation points in comparison of the regular grid presented in the interpolation.
The mean and standard deviation of this irregular decimated from the regular interpolated are $6.8386$ m and $107.7343$ m in the $x$-direction and $30.8799$ m and $28.3849$ m in the $y$-direction, respectively.
Both application were made with an Intel core i7 7700HQ@2.8GHz processor in single-processing and 
single-threading modes. 

As the study area is very large, the main magnetic field varies with position.
For this application, we set the main field direction as that of a mid location 
(latitude $-6.5^{\circ}$ and longitude $-50.75^{\circ}$) where the declination is $-19.86^{\circ}$ and
the inclination is $-7.4391^{\circ}$. Both values were calculated using the magnetic field calculator from NOAA
at 1st January, 2014 (epoch of the survey). We set the equivalent layer at $300$ meters above the ground ($600$ m below the data).
Figure \ref{fig:carajas_residuals_comparison}b shows the residuals obtained after using our method to fit
the interpolated data with mean $0.0762$ nT and the standard deviation  $0.4886$ nT, revealing an acceptable data fitting. Our method took $\approx 390.80$ seconds to converge at about $200$ iterations.
Figure \ref{fig:carajas_residuals_comparison}c shows the residuals obtained after using our method to fit
the decimated data with mean $0.0717$ nT and standard deviation 
$0.3144$ nT with a better fit than the produced by the interpolated data. In this case, our method took $\approx 385.56$ seconds to converge at about $200$ iterations (Figure \ref{fig:carajas_residuals_comparison}d). 
The convergence curve reveals a good convergence rate obtained with the decimated 
irregular grid. This result shows the robustness of our method in processing irregular grid. Notice that we used $200$ iterations in our method of the interpolated regular grid and the mean residual still decreasing up to $2000$. This happens because the invariance convergence criterion was met as the mean residuals are very small and each iteration decreases less than $0.00001$

With $1,310,000$ observation points, it would be necessary $12.49$ Terabytes of RAM to store the full
sensitivity matrix with the classical method. 
In this case, our method uses only $59.97$ Megabytes, allowing regular desktop computers to be able 
to process this amount of data.


%
%\textbf{Alternative text} $\rightarrow$
%
%As the study area is very large, the main magnetic field varies with position.
%For this application, we set the main field direction as that of a mid location 
%(latitude $-6.5^{\circ}$ and longitude $-50.75^{\circ}$) where the declination is $-19.86^{\circ}$ 
%and inclination is $XXXXXX^{\circ}$ according to IGRF model at 1st January, 2014 (epoch of the survey).
%(É ESTRANHO CALCULAR A DECLINAÇÃO COM UM MODELO E A INCLINAÇÃO COM OUTRO)
%
%$\leftarrow$ \textbf{Alternative text}
%

%(Figure \ref{fig:carajas_real_data_decimated_gridline}a), we obtain the  predicted data shown in 
%Figure \ref{fig:carajas_gz_predito_mag_gridline}a and data residuals 
%(Figure \ref{fig:carajas_gz_predito_mag_gridline}b) with mean $0.0762$ nT and the standard deviation 
%$0.4886$ nT, revealing an acceptable data fitting.
%Our method took $\approx 390.80$ seconds to converge at about $200$ iterations.
%
%POR QUE A CURVA DE CONVERGENCIA NÃO FOI MOSTRADA?
%

%By applying our method to the decimated irregular grid 
%(Figure \ref{fig:carajas_real_data_decimated_gridline}b), we obtain the predicted data shown in 
%Figure \ref{fig:carajas_gz_predito_mag_decimated}a and data residuals 
%(Figure \ref{fig:carajas_gz_predito_mag_decimated}b) with mean $0.0717$ nT and standard deviation 
%$0.3144$ nT. In this case, our method took $\approx 385.56$ seconds 
%to converge at about $200$ iterations (Figure \ref{fig:convergence_carajas_mag_decimated}).
%(TEM ALGO ESTRANHO AQUI. COMO QUE 2000 ITERAÇÕES FOI MAIS MAIS RÁPIDO DO QUE AS 200 ITERAÇÕES DO GRID INTERPOLADO?).

%The convergence curve reveals a good convergence rate obtained with the decimated 
%irregular grid. This result shows the robustness of our method in processing irregular grid. Notice that we used in our %method of the interpolated regular grid $200$ iterations and the mean residual still decreasing up to $2000$. This happens %because the invariance convergence criterion was met as the mean residuals are very small and each iteration decreases less %than $0.00001$

%We have found, in applying our method to the decimated irregular grid, that the data residual amplitude (Figure \ref{fig:carajas_gz_predito_mag_decimated}b) is lower than the data residual amplitude (Figure 
%\ref{fig:carajas_gz_predito_mag_gridline}b) obtained by applying our method to the interpolated regular grid.
%It occurs because  the  process of decimating the original irregularly data creates neither new observation points nor new data. Rather, the interpolation of the original irregularly data creates either new observation points or new data. 

Finally, Figure \ref{fig:carajas_upward_comparison}a shows the upward-continued magnetic data to a
horizontal plane located at $5, \,000$ m using the estimated equivalent layer obtained by applying our
method to the decimated irregular grid. This process took $\approx 2.64$ seconds, showing good results 
without visible errors or border effects. 
Figure \ref{fig:carajas_upward_comparison}b shows the upward-continued magnetic data to a
horizontal plane located at $5, \,000$ m using the classical Fourier filtering method to the decimated irregular grid.
This process took $\approx 0.5$ seconds. Altough, no border effect could be observed, this result showed unexpected high values instead of the expected attenuation of the anomalies.
\section{Conclusions}

We have proposed  a fast convolutional equivalent-layer technique for processing magnetic data 
whose computation time is more than four orders of magnitude less than the classical equivalent layer.
Mathematically, we have  demonstrated that the sensitivity matrix associated with the linear system of the magnetic equivalent layer carries the structure of BTTB matrices, which means not only a very low computational cost to calculate a matrix-vector product, but also the possibility to store only the first column of the matrix BCCB. 
In this work, our novel fast convolutional equivalent-layer technique  uses only one equivalent source to calculate the first six columns of the inverse of distance second derivatives matrices and to set up the first column of the BCCB matrix embbeded from the original magnetic kernel sensitivity matrix.
We solve the linear system by adapting the method of Conjugate Gradient Least Square to compute fastly the BTTB matrix-vector product of the magnetic forward modeling in the equivalent-layer technique.

The comparisons between the performances of classical equivalent-layer technique  and  fast convolutional equivalent-layer technique using synthetic magnetic data show similar estimates of the physical-property distribution over the equivalent layer.
The difference in time, however, is noticeable: $2.04$ seconds using the classical approach and $0.083$ seconds using our approach. This difference was obtained with a mid-size mesh of $100 \times 50$ points, greater results can be obtained if more observation points are used.

The comparisons between the performances of classical approach in the Fourier domain and fast convolutional equivalent-layer technique for processing synthetic magnetic data show that the computational time required by the classical Fourier approach is the lowest one.
However, the classical Fourier approach requires not only expanding the data by using a padding function to avoid the border effects but also it requires the measurement of the data on a regular grid and on a planar observation surface.
Although the fast convolutional equivalent-layer technique  also requires that the data be measure on a regular grid and the observation surface be planar, the synthetic tests show the robustness of our method to deal  either with irregular grids or with uneven observation surface.
This robustness of our method may fail if the horizontal coordinates of the observations are greatly scattered
or if the observation surface is rugged.
However, a poor performance of our method  can be easily detected  because it leads to poor data fitting and the decay of the data-misfit function along the iterations (convergence curve) does not exhibit an invariance along successive iterations. 

On an irregular grid totaling $1,310,000$ observation points, the field data over the Carajás Province, northern Brazil, would require $12.49$ Terabytes of RAM to store the full sensitivity matrix  to run the classical equivalent-layer technique. 
However, the fast convolutional equivalent-layer technique  neither requires the full computation nor the storage of a sensitivity matrix.
Taking advantage of the symmetric or skew-symmetric matrices structures, it is possible to reconstruct the whole sensitivity matrix using only $59.97$ Megabytes.
When performed on a standard laptop computer with an Intel Core i7 7700HQ@2.8GHz processor in single-processing and single-threading modes, the total time spent by our method for estimating the physical-property distribution over the equivalent layer was approximately $385.56$ seconds and for upward-continuing the total of $1,310,000$ magnetic observations  was $2.64$ seconds.


\section{Acknowledgements}

Diego Takahashi was supported by a Phd scholarship from CAPES. 
Val{\'e}ria C.F. Barbosa was supported by fellowships from CNPq (grant 307135/2014-4) 
and FAPERJ (grant 26/202.582/2019). Vanderlei C. Oliveira Jr. was supported 
by fellowships from CNPq (grant 308945/2017-4) and FAPERJ (grant E-26/202.729/2018). 
The authors thank the Geological Survey of Brazil (CPRM) for providing the field data.

\clearpage

% Tables and figures
%\renewcommand{\figdir}{Fig} % figure directory

%% Methodology
%\plot{Figure1}{width=\textwidth}{
%	{Schematic representation of an $N_{x} \times N_{y}$ regular grid of points (black dots) defined by 
%	$N_{x} = 4$ and $N_{y} = 3$. The grids are oriented along the (a) $x$-axis and (b) $y$-axis. The grid 
%	coordinates are $x_{k}$ and $y_{l}$, where the $k = 1, \dots, N_{x}$ and $l = 1, \dots, N_{y}$ are 
%	called the grid indices. The insets show the grid indices $k$ and $l$.}
%	\label{fig:methodology}
%}

\plot{schematic_regular_grids}{width=\textwidth}{
	{Schematic representation of an $N_{x} \times N_{y}$ regular grid of points (black dots) with
	$N_{x} = 3$ and $N_{y} = 2$, where each point has an associated index. This index may represent
	$i$ or $j$, that are associated with observation points $(x_{i}, y_{i}, z_{0})$ and 
	equivalent sources $(x_{j}, y_{j}, z_{c})$. Left panel shows an example of $x$-oriented grid,
	with indices varying along $x$-axis, while right panel shows an example of $y$-oriented grid, 
	with indices varying along $y$-axis.}
	\label{fig:regular-grids}
}

%\plot{4_equivalent_sources}{width=\textwidth}{
%	{Representation of the four equivalent sources (black dots) needed to reconstruct the non-symmetric matrix $\mathbf{A}$ (equation \ref{eq:aij_mag}). Each of the equivalent sources are located in the corner of the simulated regular grid of $M_x = 4$ and $M_y = 3$ sources. The influence of these sources on each of the observation points (blue dots) i the regular grid of $N_x = 4$ and $N_y = 3$ will give the four columns necessary of matrix $\mathbf{A}$.}
%	\label{fig:4_equivalent_sources}
%}
%
%%% Computational performance
%
%\plot{flops_mag}{width=\textwidth}{
%	{Number of flops necessary to estimate the parameter vector $\mathbf{\hat{p}}$ using the non-iterative classical method (equation \ref{eq:flops-normal-cholesky}) the CGLS (equation \ref{eq:flops-cgls}) and our modified CGLS method (equation \ref{eq:flops-cgls-bccb}) with $N^{it} = 50$. 
%	The observation point $N$ varied from $5,000$ to $1,000,000$. The \emph{radix-2} 2D FFT
%	algorithm was considered for our method, with $\kappa = 5$.}
%	\label{fig:flops}
%}
%
%\plot{time_comparison_mag}{width=\textwidth}{
%	{Comparison between the runtime of the equivalent-layer technique using the classical, the CGLS algorithm and our method. The values for the CGLS and our methods were obtained for $N^{it} = 50$ iterations.}
%	\label{fig:solve_time}
%}
%
%\plot{time_sources_mag}{width=\textwidth}{
%	{Comparison between the runtime to calculate the first column of the BCCB matrix embbeded from $\mathbf{A}$ (equation \ref{eq:aij_mag}) using only one and using four equivalent sources. Although the time is very similar, with one source a small advantage can be observed as the number of data $N$ increases. This test was done from $N = 10,000$ to $N = 700,000$ with increases of $5,625$ observation points.}
%	\label{fig:sources_time}
%}
%
%%% Synthetic data part I
%
%\plot{model_mag_synthetic}{width=\textwidth}{
%	{Observed synthetic magnetic field data. A regular grid of $80 \times 80$ points was used, totaling $N = 6,\, 400$ observation points. Three bodies were modeled: two prisms and a sphere with inclination, declination and intensity of $0^{\circ}$ and $45^{\circ}$ and $2\times\sqrt{2}$ A/m, respectively.
%The black lines represent the horizontal projection of the sources.}
%	\label{fig:model_mag_synthetic}
%}
%
%\plot{predicted_synthetic_mag}{width=8cm}{
%	{ Synthetic test with regular grid - (a) Predicted data using a classical linear inversion method (equation \ref{eq:estimated-p-parameter-space}). (b) Data residuals,  defined as the difference between the observed (Figure \ref{fig:model_mag_synthetic}) and the predicted data (panel a). The black lines represent the horizontal projection of the sources. The total time for the inversion was $17.6$ seconds.}
%	\label{fig:predicted_synthetic_mag}
%}
%
%\plot{predicted_bccb_mag}{width=8cm}{
%	{Synthetic test with regular grid - (a) Predicted data using our method with the fast BTTB matrix-vector product. (b) Data residuals,  defined as the difference between the observed (Figure \ref{fig:model_mag_synthetic}) and the predicted data (panel a). The black lines represent the horizontal projection of the sources. 
%The total time for the inversion was $0.18$ seconds.}
%	\label{fig:predicted_bccb_mag}
%}
%
%\plot{convergence_synthetic_mag}{width=\textwidth}{
%	{Synthetic test with regular grid - Convergence analysis of our method with the fast BTTB matrix-vector product.}
%	\label{fig:convergence_synthetic_mag}
%}
%
%%% Synthetic data part II
%
%
%\plot{model_mag_synthetic_irregular_10}{width=8cm}{
%	{ Synthetic test with irregular grid with uncertainty of $10\%$ -
%	(a) Simulated irregular grid with  $N = 5,\, 000$ observation points (dots)  built up from 
%a regular $100 \times 50$ grid in the $x-$ and $y-$directions,  with a grid spacing of
%$\Delta x$ of $101.01$ m and $\Delta y$ of $163.265$ m  along the $x$- and $y$-directions, respectively.
%The $x$- and $y$-coordinates of the observations in the irregular grid were contamined by the pseudorandom, zero-mean Gaussian noise with standart deviations of $10\%$ of $\Delta x$ and $\Delta y$ 
%in both $x$- and $y$-directions. (b) Observed synthetic magnetic field data using the irregular grid in panel a. Three bodies were modeled: two prisms and a sphere with inclination, declination and intensity of $0^{\circ}$ and $45^{\circ}$ and $2\times\sqrt{2}$ A/m, respectively. 
%The black lines represent the horizontal projection of the sources.}
%	\label{fig:model_mag_synthetic_irregular_10}
%}
%
%\plot{predicted_synthetic_mag_irregular_10}{width=8cm}{
%	{Synthetic test with irregular grid with uncertainty of $10\%$ - (a) Predicted data using a classical linear inversion method (equation \ref{eq:estimated-p-parameter-space}) for the irregular grid in Figure \ref{fig:model_mag_synthetic_irregular_10}a. (b) Data residuals,  defined as the difference between the observed (Figure \ref{fig:model_mag_synthetic_irregular_10}b) and the predicted data (panel a), with mean $0.3628$ nT and standart deviation of $0.2727$ nT.
%The black lines represent the horizontal projection of the sources.}
%	\label{fig:predicted_synthetic_mag_irregular_10}
%}
%
%\plot{predicted_bccb_mag_irregular_10}{width=8cm}{
%	{Synthetic test with irregular grid  with uncertainty of $10\%$  - (a) Predicted data using our method with the fast BTTB matrix-vector product  for the irregular grid in Figure \ref{fig:model_mag_synthetic_irregular_10}a. (b) Data residuals,  defined as the difference between the observed (Figure \ref{fig:model_mag_synthetic_irregular_10}b) and the predicted data (panel a), with mean $0.6024$ nT and standart deviation of $0.4998$ nT.
%The black lines represent the horizontal projection of the sources.}
%	\label{fig:predicted_bccb_mag_irregular_10}
%}
%
%\plot{convergence_synthetic_mag_irregular_10}{width=\textwidth}{
%	{Synthetic test with irregular grid with uncertainty of $10\%$  -  Convergence analysis of our method with the fast BTTB matrix-vector product using an irregular grid with $10\%$ of pertubation on the $x$- and $y$-coordinates as shown in Figure \ref{fig:model_mag_synthetic_irregular_10}a.}
%	\label{fig:convergence_synthetic_mag_irregular_10}
%}
%
%%
%%
%
%\plot{model_mag_synthetic_irregular_20}{width=8cm}{	
%	{(a) Synthetic magnetic field grid visualization. A irregular grid of $100 \times 50$ points was used, totaling $N = 5,\, 000$ observation points. Standart deviations of $20\%$ in the $x$-direction and $20\%$ in the $y$-direction were applied. (b) Observed synthetic magnetic field data using this irregular grid in panel a. Three bodies were modeled: two prisms and a sphere with inclination, declination and intensity of $0^{\circ}$ and $45^{\circ}$ and $2\times\sqrt{2}$ A/m, respectively.}
%	\label{fig:model_mag_synthetic_irregular_20}
%}
%
%\plot{predicted_synthetic_mag_irregular_20}{width=8cm}{	
%	{(a) Predicted data using a classical linear inversion method (equation \ref{eq:estimated-p-parameter-space}) for the irregular grid in figure \ref{fig:model_mag_synthetic_irregular_20}a. (b) Residuals between the observed (\ref{fig:model_mag_synthetic_irregular_20}b) and the predicted data (panel a), with mean $0.3630$ nT and standart deviation of $0.2731$ nT.}
%	\label{fig:predicted_synthetic_mag_irregular_20}
%}
%
%\plot{predicted_bccb_mag_irregular_20}{width=8cm}{	
%	{(a) Predicted data using the CGLS method with the fast BTTB matrix-vector product for for the irregular grid in figure \ref{fig:model_mag_synthetic_irregular_20}a. (b) Residuals between the observed (\ref{fig:model_mag_synthetic_irregular_20}b) and the predicted data (panel a), with mean $0.7147$ nT and standart deviation of $0.5622$ nT.}
%	\label{fig:predicted_bccb_mag_irregular_20}
%}
%
%\plot{convergence_synthetic_mag_irregular_20}{width=\textwidth}{	
%	{Convergence analysis of the CGLS method for the synthetic application of the magnetic equivalent layer using an irregular grid with $20\%$ of pertubation on the $x$-\textit{direction} and $y$-\textit{direction}.}
%	\label{fig:convergence_synthetic_mag_irregular_20}
%}
%
%%
%%
%
%\plot{model_mag_synthetic_irregular_30}{width=8cm}{
%	{(a) Synthetic magnetic field grid visualization. A irregular grid of $100 \times 50$ points was used, totaling $N = 5,\, 000$ observation points. Standart deviations of $30\%$ in the $x$-direction and $30\%$ in the $y$-direction were applied. (b) Observed synthetic magnetic field data using this irregular grid in panel a. Three bodies were modeled: two prisms and a sphere with inclination, declination and intensity of $0^{\circ}$ and $45^{\circ}$ and $2\times\sqrt{2}$ A/m, respectively.}
%	\label{fig:model_mag_synthetic_irregular_30}
%}
%
%\plot{predicted_synthetic_mag_irregular_30}{width=8cm}{	
%	{(a) Predicted data using a classical linear inversion method (equation \ref{eq:estimated-p-parameter-space}) for the irregular grid in figure \ref{fig:model_mag_synthetic_irregular_30}a. (b) Residuals between the observed (\ref{fig:model_mag_synthetic_irregular_30}b) and the predicted data (panel a), with mean $0.3634$ nT and standart deviation of $0.2735$ nT.}
%	\label{fig:predicted_synthetic_mag_irregular_30}
%}
%
%\plot{predicted_bccb_mag_irregular_30}{width=8cm}{	
%	{(a) Predicted data using the CGLS method with the fast BTTB matrix-vector product for for the irregular grid in figure \ref{fig:model_mag_synthetic_irregular_30}a. (b) Residuals between the observed (\ref{fig:model_mag_synthetic_irregular_30}b) and the predicted data (panel a), with mean $0.9788$ nT and standart deviation of $0.7462$ nT.}
%	\label{fig:predicted_bccb_mag_irregular_30}
%}
%
%\plot{convergence_synthetic_mag_irregular_30}{width=\textwidth}{
%	{Convergence analysis of the CGLS method for the synthetic application of the magnetic equivalent layer using an irregular grid with $30\%$ of pertubation on the $x$-\textit{direction} and $y$-\textit{direction}.}
%	\label{fig:convergence_synthetic_mag_irregular_30}
%}
%
%%
%%
%
%\plot{model_mag_synthetic_irregular_z5_undulating}{width=8cm}{ 
%	{(a) Undulating surface where the total-field anomaly was computed. A irregular grid of $100 \times 50$ points was used, totaling $N = 5,\, 000$ observation points. A standart deviation of $5\%$ in the $z$-direction was applied. (b) Observed synthetic magnetic field data using this irregular grid in panel a. Three bodies were modeled: two prisms and a sphere with inclination, declination and intensity of $0^{\circ}$ and $45^{\circ}$ and $2\times\sqrt{2}$ A/m, respectively.}
%	\label{fig:model_mag_synthetic_irregular_z5}
%}
%
%\plot{predicted_synthetic_mag_irregular_z5}{width=8cm}{
%	{(a) Predicted data using a classical linear inversion method (equation \ref{eq:estimated-p-parameter-space}) for the irregular grid in figure \ref{fig:model_mag_synthetic_irregular_z5}a. (b) Residuals between the observed (\ref{fig:model_mag_synthetic_irregular_z5}b) and the predicted data (panel a), with mean $0.3712$ nT and standart deviation of $0.2870$ nT.}
%	\label{fig:predicted_synthetic_mag_irregular_z5}
%}
%
%\plot{predicted_bccb_mag_irregular_z5}{width=8cm}{
%	{(a) Predicted data using the CGLS method with the fast BTTB matrix-vector product for for the irregular grid in figure \ref{fig:model_mag_synthetic_irregular_z5}a. (b) Residuals between the observed (\ref{fig:model_mag_synthetic_irregular_z5}b) and the predicted data (panel a), with mean $0.9542$ nT and standart deviation of $0.8943$ nT.}
%	\label{fig:predicted_bccb_mag_irregular_z5}
%}
%
%\plot{convergence_synthetic_mag_irregular_z5}{width=\textwidth}{
%	{Convergence analysis of the CGLS method for the synthetic application of the magnetic equivalent layer using an irregular grid with $5\%$ of pertubation on the $z$-\textit{direction}.}
%	\label{fig:convergence_synthetic_mag_irregular_z5}
%}
%
%%
%%
%
%
%\plot{model_mag_synthetic_irregular_z10_undulating}{width=8cm}{
%	{(a) Undulating surface where the total-field anomaly was computed. A irregular grid of $100 \times 50$ points was used, totaling $N = 5,\, 000$ observation points. A standart deviation of $10\%$ in the $z$-direction was applied. (b) Observed synthetic magnetic field data using this irregular grid in panel a. Three bodies were modeled: two prisms and a sphere with inclination, declination and intensity of $0^{\circ}$ and $45^{\circ}$ and $2\times\sqrt{2}$ A/m, respectively.}
%	\label{fig:model_mag_synthetic_irregular_z10}
%}
%
%\plot{predicted_synthetic_mag_irregular_z10}{width=8cm}{
%	{(a) Predicted data using a classical linear inversion method (equation \ref{eq:estimated-p-parameter-space}) for the irregular grid in figure \ref{fig:model_mag_synthetic_irregular_z10}a. (b) Residuals between the observed (\ref{fig:model_mag_synthetic_irregular_z10}b) and the predicted data (panel a), with mean $0.3865$ nT and standart deviation of $0.3216$ nT.}
%	\label{fig:predicted_synthetic_mag_irregular_z10}
%}
%
%\plot{predicted_bccb_mag_irregular_z10}{width=8cm}{
%	{(a) Predicted data using the CGLS method with the fast BTTB matrix-vector product for for the irregular grid in figure \ref{fig:model_mag_synthetic_irregular_z10}a. (b) Residuals between the observed (\ref{fig:model_mag_synthetic_irregular_z10}b) and the predicted data (panel a), with mean $1.6105$ nT and standart deviation of $1.6231$ nT.}
%	\label{fig:predicted_bccb_mag_irregular_z10}
%}
%
%\plot{convergence_synthetic_mag_irregular_z10}{width=\textwidth}{
%	{Convergence analysis of the CGLS method for the synthetic application of the magnetic equivalent layer using an irregular grid with $10\%$ of pertubation on the $z$-\textit{direction}.}
%	\label{fig:convergence_synthetic_mag_irregular_z10}
%}
%
%%
%%
%
%\plot{model_mag_synthetic_irregular_z20_undulating}{width=8cm}{
%	{(a) Undulating surface where the total-field anomaly was computed. A irregular grid of $100 \times 50$ points was used, totaling $N = 5,\, 000$ observation points. A standart deviation of $20\%$ in the $z$-direction was applied. (b) Observed synthetic magnetic field data using this irregular grid in panel a. Three bodies were modeled: two prisms and a sphere with inclination, declination and intensity of $0^{\circ}$ and $45^{\circ}$ and $2\times\sqrt{2}$ A/m, respectively.}
%	\label{fig:model_mag_synthetic_irregular_z20}
%}
%
%\plot{predicted_synthetic_mag_irregular_z20}{width=8cm}{
%	{(a) Predicted data using a classical linear inversion method (equation \ref{eq:estimated-p-parameter-space}) for the irregular grid in figure \ref{fig:model_mag_synthetic_irregular_z20}a. (b) Residuals between the observed (\ref{fig:model_mag_synthetic_irregular_z20}b) and the predicted data (panel a), with mean $0.4155$ nT and standart deviation of $0.4005$ nT.}
%	\label{fig:predicted_synthetic_mag_irregular_z20}
%}
%
%\plot{predicted_bccb_mag_irregular_z20}{width=8cm}{
%	{(a) Predicted data using the CGLS method with the fast BTTB matrix-vector product for for the irregular grid in figure \ref{fig:model_mag_synthetic_irregular_z20}a. (b) Residuals between the observed (\ref{fig:model_mag_synthetic_irregular_z20}b) and the predicted data (panel a), with mean $6.6220$ nT and standart deviation of $5.901$ nT.}
%	\label{fig:predicted_bccb_mag_irregular_z20}
%}
%
%\plot{convergence_synthetic_mag_irregular_z20}{width=\textwidth}{
%	{Convergence analysis of the CGLS method for the synthetic application of the magnetic equivalent layer using an irregular grid with $20\%$ of pertubation on the $z$-\textit{direction}.}
%	\label{fig:convergence_synthetic_mag_irregular_z20}
%}
%
%%% Field Data
%
%\plot{carajas_real_data_mag}{width=\textwidth}{
%	{Observed magnetic field data of the Carajás, Brazil area. The aeromagnetic survey was done with $131$ N-S lines at approximately $-900 m$ height, totaling $N = 6,081,345$ observation points.}
%	\label{fig:carajas_real_data_mag}
%}
%
%\plot{carajas_real_data_decimated_gridline}{width=8cm}{
%	{(a) Observed magnetic field data of the Carajás, Brazil area, interpolated for a regular grid of $10,000 \times 131$, totaling $N = 1,310,000$ observation points. (b) Observed magnetic field data of the Carajás, Brazil area, decimated from the flight lines resulting in an irregular grid of $10,000 \times 131$, also totaling $N = 1,310,000$ observation points.}
%	\label{fig:carajas_real_data_decimated_gridline}
%}
%
%\plot{carajas_tf_predicted_gridline}{width=8cm}{
%	{(a) Predicted data using our method for the interpolated $10,000 \times 131$ regular grid. (b) Residuals between the observed (\ref{fig:carajas_real_data_decimated_gridline}) and the predicted data (panel a), with a mean of $0.07979$ nT and standart deviation of $0.5060$ nT.}
%	\label{fig:carajas_gz_predito_mag_gridline}
%}
%
%\plot{carajas_tf_predicted_decimated}{width=8cm}{
%	{(a) Predicted data using our method for the decimated $10,000 \times 131$ irregular grid. (b) Residuals between the observed (\ref{fig:carajas_real_data_decimated_gridline}) and the predicted data (panel b), with a mean of $0.07348$ nT and standart deviation of $0.3172$ nT.}
%	\label{fig:carajas_gz_predito_mag_decimated}
%}
%
%\plot{convergence_carajas_mag_decimated}{width=\textwidth}{
%	{Convergence analysis of the CGLS method for the field data of Carajás, Brazil using the magnetic equivalent layer with a decimated irregular grid of $10,000 \times 131$ observation points up to 2,000 iterations.}
%	\label{fig:convergence_carajas_mag_decimated}
%}
%
%\plot{up5000_carajas_mag_decimated}{width=\textwidth}{
%	{Upward continuation transformation of real data of Carajás, Brazil at $5,000$ meter. It was necessary $2.64$ seconds to complete the process.}
%	\label{fig:up5000_carajas_decimated_mag}
%}

\clearpage

% Appendices
%\appendix
\append{Flops computations}

%======================================================================================
\subsection{Classical flops count}
%======================================================================================

The flops count of the classical approach to solve the linear system (equation \ref{eq:estimated-p-parameter-space}) using the Cholesky factorization is given by equation \ref{eq:flops-normal-cholesky}. The step-by-step count follows:
\begin{itemize}
\item[\textbf{(1)}] $\mathbf{A}^{\top}\mathbf{A}$: $2 N^3$ (one matrix-matrix product).

\item[\textbf{(2)}] $\mathbf{A}^{\top} \mathbf{A}$: $\dfrac{1}{3} N^3$ (one Cholesky factorization $\mathbf{C_f}$).

\item[\textbf{(3)}] $\mathbf{A}^{\top} \mathbf{d}^{o}$: $2 N^2$ (one matrix-vector product).

\item[\textbf{(4)}] $\mathbf{C_f} (\mathbf{A}^{\top} \mathbf{d}^{o})$: $2 N^2$ (one matrix-vector product).

\item[\textbf{(5)}] $\mathbf{C_f}^{\top} (\mathbf{C_f} \mathbf{A}^{\top} \mathbf{d}^{o})$: $2 N^2$ (one matrix-vector product).
\end{itemize}
Summing all calculations: 
\begin{equation}
f_{classical} =  \dfrac{7}{3} N^{3} + 6 N^{2}\: ,
\label{eq:flops-normal-cholesky-append}
\end{equation}

%======================================================================================
\subsection{CGLS flops count}
%======================================================================================

The flops count of CGLS algorithm \ref{al:cgls-algorithm} can be summarized as:

Out of the loop:

\begin{itemize}
%\item[\textbf{(1)}] $\mathbf{d}^{o} - \mathbf{A} \hat{\mathbf{p}}^{(0)}$: $2 N^2 + N$ (one matrix-vector product %and one vector subtraction)

\item[\textbf{(1)}] $\mathbf{A}^{\top} \mathbf{s}$: $2 N^2$ (one matrix-vector product).
\end{itemize}

Inside the loop:

\begin{itemize}
\item[\textbf{(1)}] $\dfrac{{\mathbf{r}^{(it)}}^{\top} \, \mathbf{r}^{(it)}} {{\mathbf{r}^{(it - 1)}}^{\top} \, \mathbf{r}^{(it - 1)}}$: $4 N$ (two vector-vector products).

\item[\textbf{(2)}] $\mathbf{r}^{it} - \alpha_{it} \,\beta_{it} \, \mathbf{c}^{(it - 1)}$: $2 N$ (one scalar-vector product and one vector subtraction).

\item[\textbf{(3)}] $\frac{{||\mathbf{r}^{(it)}||_2}^2}{({\mathbf{c}^{(it)}}^{\top} \, \mathbf{A}^{\top})(\mathbf{A} \, \mathbf{c}^{(it)})}$: $2 N^2 + 2N$ (one matrix-vector and one vector-vector product).

\item[\textbf{(4)}] $\hat{\mathbf{p}}^{it} - \alpha_{it} \, \mathbf{c}^{(it)}$: $2 N$ (one vector subtraction).

\item[\textbf{(5)}] $\mathbf{s}^{it} - \alpha_{it} \, \mathbf{A} \, \mathbf{c}^{(it)}$: $2 N$ (one vector subtraction, the matrix-vector product was calculated in step 3).

\item[\textbf{(6)}] $ \mathbf{A}^{\top} \, \mathbf{s}^{(it + 1)}$: $2 N^2$ (one matrix-vector product).
\end{itemize}
The result of all flops count leads to:
\begin{equation}
f_{cgls} =  2 N^{2} + it \, (4 N^{2} + 12 N) \: .
\label{eq:flops-cgls-append}
\end{equation}

%======================================================================================
\subsection{Our modified CGLS flops count}
%======================================================================================

All the flops count presented in previous section for the CGLS remains, only substituting the  out of the loop matrix-vector product in step 1 and the two matrix-vector products inside the loop in steps 3 and 6.
The computations necessary to carry the matrix-vector used in this work are given by:

\begin{itemize}
\item[\textbf{(1)}] $\mathbf{L}$: $\kappa  \, 4 N \log_2(4N)$ (one 2D FFT for the eigenvalues calculation of the sensitivity matrix $\mathbf{A}$ or the transposed sensitivity matrix $\mathbf{A}^{\top}$).

\item[\textbf{(2)}] $\mathbf{F}_{2Q} \, \mathbf{V} \, \mathbf{F}_{2P}$: $\kappa  \, 4 N \log_2(4N)$ (one 2D FFT).

\item[\textbf{(3)}] $\mathbf{L} \circ \left(\mathbf{F}_{2Q} \, \mathbf{V} \, \mathbf{F}_{2P} \right)$: $24 N$ (one complex Hadamard matrix multiplication).

\item[\textbf{(4)}] $\mathbf{F}_{2Q}^{\ast} \left[ 
\mathbf{L} \circ \left(\mathbf{F}_{2Q} \, \mathbf{V} \, \mathbf{F}_{2P} \right) 
\right] \mathbf{F}_{2P}^{\ast}$: $\kappa  \, 4 N \log_2(4N)$ (one inverse 2D FFT).
\end{itemize}
Matrix-vector product total:  $\kappa  \, 12 N \log_2(4 N) + 24 N$.

As matrix $\mathbf{A}$ (equation \ref{eq:aij_mag}) and its transposed never changes, it is not necessary to calculate the eigenvalues inside the loop at each iteration, we are considering that both are calculated out of the loop. Inside the loop, steps 2 to 4 are repeated two times per iteration. Substituting this result into the CGLS flops count (equation \ref{eq:flops-cgls-append}) leads to:
\begin{equation}
f_{ours} =  \kappa  \, 16 N \log_2(4 N) + 24 N + it \, (\kappa  \, 16 N \log_2 (4 N) + 60 N).
\label{eq:flops-cgls-bccb-append}
\end{equation}

%\append{Matrix $\mathbf{C}$}


This appendix illustrates the matrix $\mathbf{C}$ (equation \ref{eq:w_Cv}) 
obtained with the $x$- and $y$-oriented grids illustrated in Figure \ref{fig:methodology}
and also presents some of its relevant properties.

Matrix $\mathbf{C}$ (equation \ref{eq:w_Cv})
is circulant blockwise, formed by $2Q \times 2Q$ blocks, where
each block $\mathbf{C}_{q}$, $q = 0, \dots, Q-1$, is a $2P \times 2P$ circulant matrix. 
Similarly to the BTTB matrix $\mathbf{A}$ (equations \ref{eq:BTTB_A} and 
\ref{eq:A-x-oriented-example}--\ref{eq:Aq-y-oriented}), the index $q$ 
varies from $0$ to $Q - 1$. Additionally, the blocks lying 
above the main diagonal are equal to those located below.

It is well-known that a circulant matrix can be defined by properly downshifting 
its first column \citep[][ p. 206]{vanloan1992}. Hence, the BCCB matrix $\mathbf{C}$ 
(equation \ref{eq:w_Cv}) can be obtained from its 
first column of blocks, which is given by
\begin{equation}
\left[\mathbf{C} \right]_{(0)} = 
\begin{bmatrix}
\mathbf{C}_{0} \\
\vdots \\
\mathbf{C}_{Q-1} \\
\mathbf{0} \\
\mathbf{C}_{Q-1} \\
\vdots \\
\mathbf{C}_{1}
\end{bmatrix}_{4N \times 2P} \: ,
\label{eq:C-first-column-blocks}
\end{equation}
where $\mathbf{0}$ is a $2P \times 2P$ matrix of zeros. Similarly, each block 
$\mathbf{C}_{q}$, $q = 0, \dots, Q-1$, can be obtained by downshifting its first 
column
\begin{equation}
\mathbf{c}^{q}_{0} = 
\begin{bmatrix}
a^{q}_{0} \\
\vdots \\
a^{q}_{P-1} \\
0 \\
a^{q}_{P-1} \\
\vdots \\
a^{q}_{1}
\end{bmatrix}_{2P \times 1} \: ,
\label{eq:Cq-first-column}
\end{equation}
where $a^{q}_{p}$ (equation \ref{eq:aqp_equiv_aij}), $p = 0, \dots, P-1$, are the elements 
forming the block $\mathbf{A}_{q}$ (equations \ref{eq:Aq_block} and 
\ref{eq:A-x-oriented-example}--\ref{eq:Aq-y-oriented}).
The downshift can be thought off as permutation that pushes the components of a column vector 
down one notch with wraparound \citep[][ p. 20]{golub-vanloan2013}.
To illustrate this operation, consider our $y$-oriented grid illustrated in Figure \ref{fig:methodology}b. 
In this case, the resulting 
BCCB matrix $\mathbf{C}$ (equation \ref{eq:w_Cv}) is given by 
\begin{equation}
\mathbf{C} =
\begin{bmatrix}
\mathbf{C_{0}} & \mathbf{C_{1}} & \mathbf{C_{2}} & \mathbf{C_{3}} & \mathbf{0}     & \mathbf{C_{3}} & \mathbf{C_{2}} & \mathbf{C_{1}} \\
\mathbf{C_{1}} & \mathbf{C_{0}} & \mathbf{C_{1}} & \mathbf{C_{2}} & \mathbf{C_{3}} & \mathbf{0}     & \mathbf{C_{3}} & \mathbf{C_{2}} \\
\mathbf{C_{2}} & \mathbf{C_{1}} & \mathbf{C_{0}} & \mathbf{C_{1}} & \mathbf{C_{2}} & \mathbf{C_{3}} & \mathbf{0}     & \mathbf{C_{3}} \\
\mathbf{C_{3}} & \mathbf{C_{2}} & \mathbf{C_{1}} & \mathbf{C_{0}} & \mathbf{C_{1}} & \mathbf{C_{2}} & \mathbf{C_{3}} & \mathbf{0}     \\
\mathbf{0}     & \mathbf{C_{3}} & \mathbf{C_{2}} & \mathbf{C_{1}} & \mathbf{C_{0}} & \mathbf{C_{1}} & \mathbf{C_{2}} & \mathbf{C_{3}} \\
\mathbf{C_{3}} & \mathbf{0}     & \mathbf{C_{3}} & \mathbf{C_{2}} & \mathbf{C_{1}} & \mathbf{C_{0}} & \mathbf{C_{1}} & \mathbf{C_{2}} \\
\mathbf{C_{2}} & \mathbf{C_{3}} & \mathbf{0}     & \mathbf{C_{3}} & \mathbf{C_{2}} & \mathbf{C_{1}} & \mathbf{C_{0}} & \mathbf{C_{1}} \\
\mathbf{C_{1}} & \mathbf{C_{2}} & \mathbf{C_{3}} & \mathbf{0}     & \mathbf{C_{3}} & \mathbf{C_{2}} & \mathbf{C_{1}} & \mathbf{C_{0}}
\end{bmatrix}_{4N \times 4N},
\label{eq:C-y-oriented}
\end{equation}
where each block $\mathbf{C}_{q}$, $q = 0, 1, 2, 3$, is represented as follows 
\begin{equation}
\tensor{C}_{q} =
\begin{bmatrix}
a^{q}_{0} & a^{q}_{1} & a^{q}_{2} & 0         & a^{q}_{2} & a^{q}_{1} \\
a^{q}_{1} & a^{q}_{0} & a^{q}_{1} & a^{q}_{2} & 0         & a^{q}_{2} \\
a^{q}_{2} & a^{q}_{1} & a^{q}_{0} & a^{q}_{1} & a^{q}_{2} & 0         \\
0         & a^{q}_{2} & a^{q}_{1} & a^{q}_{0} & a^{q}_{1} & a^{q}_{2} \\
a^{q}_{2} & 0         & a^{q}_{2} & a^{q}_{0} & a^{q}_{0} & a^{q}_{1} \\
a^{q}_{1} & a^{q}_{2} & 0         & a^{q}_{2} & a^{q}_{1} & a^{q}_{0}
\end{bmatrix}_{2P \times 2P}
\label{eq:Cq-y-oriented}
\end{equation}
in terms of the block elements $a^{q}_{p}$ (equation \ref{eq:aqp_equiv_aij}).
Similar matrices are obtained for our $x$-oriented grid illustrated in Figure \ref{fig:methodology}a.

BCCB matrices are diagonalized by the 2D unitary DFT 
\citep[][ p. 185]{davis1979}. It means that $\mathbf{C}$ (equation \ref{eq:w_Cv}) 
satisfies 
\begin{equation}
\mathbf{C} = 
\left(\mathbf{F}_{2Q} \otimes \mathbf{F}_{2P} \right)^{\ast} 
\boldsymbol{\Lambda}
\left(\mathbf{F}_{2Q} \otimes \mathbf{F}_{2P} \right) \: ,
\label{eq:C-diagonalized}
\end{equation}
where the symbol ``$\otimes$" denotes the Kronecker product \citep{neudecker1969},
$\mathbf{F}_{2Q}$ and $\mathbf{F}_{2P}$ are the $2Q \times 2Q$ and $2P \times 2P$ 
unitary DFT matrices \citep[][ p. 31]{davis1979}, respectively, the superscritpt 
``$\ast$" denotes the complex conjugate and $\boldsymbol{\Lambda}$ is a 
$4QP \times 4QP$ diagonal matrix containing the eigenvalues of $\mathbf{C}$.
%\append{Computations with the 2D DFT}


In the present Appendix, we deduce equation \ref{eq:DFT-system}
by using the row-ordered $vec$-operator (here designated simply as $vec$-operator).
This equation can be efficiently computed by using the 2D 
fast Fourier Transform. 
This operator was implicitly used by \citet[][ p. 31]{jain1989} to 
show the relationship between Kronecker products and separable 
transformations. The $vec$-operator defined here 
transforms a matrix into a column vector by stacking its rows. 

Let $\mathbf{M}$ be an arbitrary $N \times M$ matrix given by:
\begin{equation}
\mathbf{M} = \begin{bmatrix}
\mathbf{m}^{\top}_{1} \\ 
\vdots \\
\mathbf{m}^{\top}_{N}
\end{bmatrix} \: ,
\label{eq:matrix-M}
\end{equation}
where $\mathbf{m}_{i}$, $i = 1, \dots, N$, are $M \times 1$ vectors containing 
the rows of $\mathbf{M}$.
The elements of this matrix can be rearranged into a column vector by using the
$vec$-operator \citep[][ p. 31]{jain1989} as follows:
\begin{equation}
vec \left( \mathbf{M} \right) = \begin{bmatrix}
\mathbf{m}_{1} \\
\vdots \\
\mathbf{m}_{N}
\end{bmatrix}_{NM \times 1} \: .
\label{eq:vec-operator}
\end{equation}
This rearrangement is known as lexicographic ordering \citep[][ p. 150]{jain1989}.

Two important properties of the $vec$-operator (equation \ref{eq:vec-operator}) 
are necessary to us. 
To define the first one, consider an 
$N \times M$ matrix $\mathbf{H}$ given by
\begin{equation}
\mathbf{H} = \mathbf{P} \circ \mathbf{Q} \: ,
\label{eq:matrix-H}
\end{equation}
where $\mathbf{P}$ and $\mathbf{Q}$ are arbitrary $N \times M$ matrices and 
``$\circ$" represents the Hadamard product \citep[][ p. 298]{horn_johnson1991}.
By applying the $vec$-operator to $\mathbf{H}$ (equation \ref{eq:matrix-H}), 
it can be shown that
\begin{equation}
vec \left( \mathbf{H} \right) = 
vec \left( \mathbf{P} \right) \circ vec \left( \mathbf{Q} \right) \: .
\label{eq:vec-matrix-H}
\end{equation}
To define the second important property of $vec$-operator, 
consider an $N \times M$ matrix $\mathbf{S}$ defined by 
the separable transformation \citet[][ p. 31]{jain1989}:
\begin{equation}
\mathbf{S} = \mathbf{P \, M \, Q} \: ,
\label{eq:matrix-S}
\end{equation}
where $\mathbf{P}$ and $\mathbf{Q}$ are arbitrary $N \times N$ and $M \times M$ 
matrices, respectively.
By implicitly applying the $vec$-operator to 
the $\mathbf{S}$ (equation \ref{eq:matrix-S}), 
\citet[][ p. 31]{jain1989} show that:
\begin{equation}
vec \left( \mathbf{S} \right) = 
\left( \mathbf{P} \otimes \mathbf{Q}^{\top} \right) 
vec \left( \mathbf{M} \right) \: ,
\label{eq:vec-matrix-S}
\end{equation}
where ``$\otimes$" denotes the Kronecker product \citep{neudecker1969}.
It is important to stress the difference between equation \ref{eq:vec-matrix-S}
and that presented by \citet{neudecker1969}, which is more commonly found in 
the literature.
While that equation uses a $vec$-operator that transforms a matrix into a column 
vector by stacking its columns, equation \ref{eq:vec-matrix-S} 
uses the $vec$-operator defined by equation \ref{eq:vec-operator}, which 
transforms a matrix into a column vector by stacking its rows.

Now, let us deduce equation \ref{eq:DFT-system} by 
using the above-defined properties (equation \ref{eq:vec-matrix-H}
and \ref{eq:vec-matrix-S}).
We start calling attention to the right side of equation \ref{eq:vec-DFT-system}.
Consider that vector $\mathbf{w}$ (equation \ref{eq:vec-DFT-system}) 
is obtained by applying the $vec$-operator (equation \ref{eq:vec-operator}) to a matrix 
$\mathbf{W}$, whose 2D DFT $\tilde{\mathbf{W}}$ is represented by the 
following separable transformation \citep[][ p. 146]{jain1989}:
\begin{equation}
\tilde{\mathbf{W}} = \mathbf{F}_{2Q} \, \mathbf{W} \, \mathbf{F}_{2P} \: ,
\label{eq:2D-DFT-W}
\end{equation}
where $\mathbf{F}_{2Q}$ and $\mathbf{F}_{2P}$ are the $2Q \times 2Q$ and $2P \times 2P$ 
unitary DFT matrices. 
Using equation \ref{eq:vec-matrix-S} and the symmetry of unitary DFT 
matrices, we rewrite the right side of equation \ref{eq:vec-DFT-system} 
as follows:
\begin{equation}
vec \left( \tilde{\mathbf{W}} \right) = 
\left( \mathbf{F}_{2Q} \otimes \mathbf{F}_{2P} \right) 
vec \left( \mathbf{W} \right) \: .
\label{eq:right_side_DFT_system_1}
\end{equation}
Similarly, consider that $\mathbf{v}$ (equation \ref{eq:vec-DFT-system}) 
is obtained by applying the $vec$-operator (equation \ref{eq:vec-operator}) to a matrix 
$\mathbf{V}$, whose 2D DFT (equation \ref{eq:2D-DFT-W}) is 
represented by $\tilde{\mathbf{V}}$. Using equation \ref{eq:vec-matrix-S} and the symmetry 
of unitary DFT matrices, we can rewrite the 
left side of equation \ref{eq:vec-DFT-system} as follows:
\begin{equation}
\boldsymbol{\Lambda} \, vec \left( \tilde{\mathbf{V}} \right) = 
\boldsymbol{\Lambda}
\left( \mathbf{F}_{2Q} \otimes \mathbf{F}_{2P} \right) 
vec \left( \mathbf{V} \right) \: .
\label{eq:left_side_DFT_system_1}
\end{equation}
Note that both sides of equation \ref{eq:left_side_DFT_system_1}
are defined as the product of the diagonal matrix $\boldsymbol{\Lambda}$ (equation \ref{eq:C-diagonalized}) 
and a vector. In this case, the matrix-vector product can be conveniently replaced by
\begin{equation}
\boldsymbol{\lambda} \circ vec \left( \tilde{\mathbf{V}} \right) = 
\boldsymbol{\lambda} \circ
\left( \mathbf{F}_{2Q} \otimes \mathbf{F}_{2P} \right) 
vec \left( \mathbf{V} \right) \: ,
\label{eq:left_side_DFT_system_2}
\end{equation}
where $\boldsymbol{\lambda}$ is a $4QP \times 1$ vector containing the diagonal of 
$\boldsymbol{\Lambda}$ (equation \ref{eq:C-diagonalized}).
Then, consider that $\boldsymbol{\lambda}$ is obtained by applying the $vec$-operator 
(equation \ref{eq:vec-operator}) to a $2Q \times 2P$ matrix $\mathbf{L}$, we can use 
equations \ref{eq:vec-matrix-H} and \ref{eq:vec-matrix-S} to rewrite equation 
\ref{eq:left_side_DFT_system_2} as follows:
\begin{equation}
vec \left( \mathbf{L} \circ \tilde{\mathbf{V}} \right) = 
vec \left[ \mathbf{L} \circ 
\left( \mathbf{F}_{2Q} \, \mathbf{V} \, \mathbf{F}_{2P} \right) 
\right] \: .
\label{eq:left_side_DFT_system_3}
\end{equation}
Equations \ref{eq:2D-DFT-W}, \ref{eq:right_side_DFT_system_1} and 
\ref{eq:left_side_DFT_system_3} show that equation \ref{eq:vec-DFT-system}
is obtained by applying the $vec$-operator to 
\begin{equation}
\mathbf{L} \circ \left( \mathbf{F}_{2Q} \, \mathbf{V} \, \mathbf{F}_{2P} \right) = 
\mathbf{F}_{2Q} \, \mathbf{W} \, \mathbf{F}_{2P} \: .
\label{eq:DFT-system-preliminary}
\end{equation}
Finally, we premultiply both sides of equation \ref{eq:DFT-system-preliminary} by 
$\mathbf{F}_{2Q}^{\ast}$ and then postmultiply both sides of the result by 
$\mathbf{F}_{2P}^{\ast}$ to deduce equation \ref{eq:DFT-system}.
%\append{The eigenvalues of $\mathbf{C}$}


In the present Appendix, we show how to efficiently compute matrix $\mathbf{L}$
(equations \ref{eq:left_side_DFT_system_3}, \ref{eq:DFT-system-preliminary} 
and \ref{eq:DFT-system}) by using only the first column of the BCCB matrix
$\mathbf{C}$ (equation \ref{eq:w_Cv}).

We need first premultiply both sides of equation \ref{eq:C-diagonalized}
by $\left(\mathbf{F}_{2Q} \otimes \mathbf{F}_{2P} \right)$ to obtain
\begin{equation}
\left(\mathbf{F}_{2Q} \otimes \mathbf{F}_{2P} \right) \mathbf{C} = 
\boldsymbol{\Lambda}
\left(\mathbf{F}_{2Q} \otimes \mathbf{F}_{2P} \right) \: .
\label{eq:C-diagonalized2}
\end{equation}
From equation \ref{eq:C-diagonalized2}, we can easily show that 
\citep[][ p. 77]{chan-jin2007}:
\begin{equation}
	\left(\mathbf{F}_{2Q} \otimes \mathbf{F}_{2P} \right) \, 
	\mathbf{c}_{0} = \frac{1}{\sqrt{4QP}} \, \boldsymbol{\lambda} \: ,
	\label{eq:DFT_C_column}
\end{equation}
where $\mathbf{c}_{0}$ is a $4QP \times 1$ vector representing the first column of 
$\mathbf{C}$ (equation \ref{eq:w_Cv}) and 
$\boldsymbol{\lambda}$ (equation \ref{eq:left_side_DFT_system_2}) is the $4QP \times 1$ 
vector that contains the diagonal of matrix $\boldsymbol{\Lambda}$ (equation \ref{eq:C-diagonalized}) 
and is obtained by applying the $vec$-operator (equation \ref{eq:vec-operator}) to matrix $\mathbf{L}$.
Now, let us conveniently consider that $\mathbf{c}_{0}$ is obtained by applying the $vec$-operator 
to a $2Q \times 2P$ matrix $\mathbf{G}$.
Using this matrix, the property of the $vec$-operator for separable transformations 
(equation \ref{eq:matrix-S}) and the symmetry of unitary DFT matrices, equation \ref{eq:DFT_C_column} 
can be rewritten as follows
\begin{eqnarray}
	\mathbf{F}_{2Q} \, \mathbf{G} \, \mathbf{F}_{2P} = 
	\frac{1}{\sqrt{4QP}} \, \mathbf{L} \: .
	\label{eq:DFT_G}
\end{eqnarray}
This equation shows that the eigenvalues of the BCCB matrix $\mathbf{C}$ 
(equation \ref{eq:w_Cv}), forming the rows of $\mathbf{L}$,
are obtained by computing the 2D DFT of matrix $\mathbf{G}$,
which contains the elements forming the first column of the BCCB matrix 
$\mathbf{C}$ (equation \ref{eq:w_Cv}).

\newpage

\bibliographystyle{seg}  % style file is seg.bst
\bibliography{references}

\clearpage

% Tables and figures
\tabl{RAM-usage}{This table shows the RAM memory usage (in Megabytes) for storing the whole matrix $\mathbf{A}$ (equation \ref{eq:aij_mag}), the sum of all six first columns of the BCCB matrices embedded from the components of the matrix $\mathbf{H}$ from equation \ref{eq:Hi} (both need 8 bytes per element) and the matrix $\mathbf{L}$ containing the eigenvalues complex numbers (16 bytes per element) resulting from the diagonalization of matrix $\mathbf{C}$ (equation \ref{eq:w_Cv}). Here we must consider that $N$ observation points forms a $N \times N$ matrix.
\label{tab:RAM-usage}}
{
	\begin{center}
		\begin{tabular}[]{|l|c|c|c|}
			\hline
			\textbf{$N$} & \textbf{Matrix $\mathbf{A}$} & \textbf{All six first columns of BCCB matrices} & \textbf{Matrix $\mathbf{L}$}\\
			\hline 
			$100$ & 0.0763 & 0.0183 & 0.00610\\
			\hline
			$400$ & 1.22 & 0.0744 & 0.0248\\
			\hline
			$2,500$ & 48 & 0.458 & 0.1528\\
			\hline
			$10,000$ & 763 & 1.831 & 0.6104\\
			\hline
			$40,000$ & 12,207 & 7.32 & 2.4416 \\
			\hline
			$250,000$ & 476,837 & 45.768 & 15.3 \\
			\hline
			$500,000$ & 1,907,349 & 91.56 & 30.518 \\
			\hline
			$1,000,000$ & 7,629,395 & 183.096 & 61.035 \\
			\hline
		\end{tabular}
	\end{center} 
}


\tabl{mean_std}{This table shows the means and standard deviations for each residual presented in the synthetic data tests between the observation data in comparison to the classical solution and our solution.
\label{tab:mean_std}}
{
	\begin{center}
		\begin{tabular}[]{|l|c|c|}
			\hline
			Residual                                                      & Mean   & Std \\
			\hline 
			Noiseless grid classical solution                             & 0.3627 & 0.2724 \\
			\hline
			Noiseless grid our solution                                   & 0.5223 & 0.4323 \\
			\hline
			X, Y - 20\% noise grid classical solution                     & 0.3630 & 0.2731 \\
			\hline
			X, Y - 20\% noise grid our solution                           & 0.7147 & 0.5622 \\
			\hline
			X, Y - 30\% noise grid classical solution                     & 0.3634 & 0.2735 \\
			\hline
			X, Y - 30\% noise grid our solution                           & 0.9788 & 0.7462 \\
			\hline
			Z - 05\% noise surface classical solution                     & 0.3712 & 0.2870 \\
			\hline
			Z - 05\% noise surface our solution                           & 0.9542 & 0.8943 \\
			\hline
			Z - 10\% noise surface classical solution                     & 0.3865 & 0.3216 \\
			\hline
			Z - 10\% noise surface our solution                           & 1.6109 & 1.6231 \\
			\hline
		\end{tabular}
	\end{center} 
}

\tabl{mean_std_upward}{This table shows the means and standard deviations for each residual presented in the synthetic data tests section for the upward continuation transformation comparison.
	\label{tab:mean_std_upward}}
{
	\begin{center}
		\begin{tabular}[]{|l|c|c|}
			\hline
			Residual                                                      & Mean   & Std \\
			\hline
			Noiseless grid upward continuation classical solution         & 0.5130 & 0.6533 \\
			\hline
			Noiseless grid upward continuation our solution               & 0.6127 & 0.8121 \\
			\hline
			Noiseless grid upward continuation Fourier solution           & 2.1120 & 2.2870 \\
			\hline
			X, Y - 20\% noise grid upward continuation classical solution & 0.5060 & 0.6427 \\
			\hline
			X, Y - 20\% noise grid upward continuation our solution       & 0.6764 & 0.8147 \\
			\hline
			X, Y - 20\% noise grid upward continuation Fourier solution   & 2.1025 & 2.2774 \\
			\hline
			X, Y - 30\% noise grid upward continuation classical solution & 0.4988 & 0.6324 \\
			\hline
			X, Y - 30\% noise grid upward continuation our solution       & 0.7477 & 0.7796 \\
			\hline
			X, Y - 30\% noise grid upward continuation Fourier solution   & 2.1116 & 2.2763 \\
			\hline
			Z - 05\% noise surface upward continuation classical solution & 0.4979 & 0.6380 \\
			\hline
			Z - 05\% noise surface upward continuation our solution       & 1.5656 & 0.9857 \\
			\hline
			Z - 05\% noise surface upward continuation Fourier solution   & 2.0925 & 2.2907 \\
			\hline
			Z - 10\% noise surface upward continuation classical solution & 0.4792 & 0.6234 \\
			\hline
			Z - 10\% noise surface upward continuation our solution       & 4.2127 & 2.3897 \\
			\hline
			Z - 10\% noise surface upward continuation Fourier solution   & 2.0728 & 2.2924 \\
			\hline
		\end{tabular}
	\end{center} 
}

\tabl{mean_std_real_data}{This table shows the means and standard deviations for the residuals presented in the real data tests section, for the solution of the regular interpolated grid and the irregular decimated grid.
	\label{tab:mean_std_real_data}}
{
	\begin{center}
		\begin{tabular}[]{|l|c|c|}
			\hline
			Residual                               & Mean   & Std \\
			\hline
			Regular interpolated grid              & 0.0735 & 0.3172 \\
			\hline
			Irregular decimated grid               & 0.0798 & 0.5060 \\
			\hline
		\end{tabular}
	\end{center} 
}


\end{document}
