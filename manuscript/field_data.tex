\section{Application to field data}

The field data application was performed with the aeromagnetic data of Carajás, Pará, Brazil, provided by CPRM.
The survey is composed of $131$ flight lines N-S oriented, spacing $\Delta y = 3,000$ m. The magnetometer (Scintrex CS-3) was set to a interval between measurements of $0.1$ s giving a spacing $\Delta x = 7.65$ m. The average flight heigth is $\Delta z = -900$ m. The total number of observation points is $N = 6,081,345$. Figure \ref{fig:carajas_real_data_mag} shows the observed magnetic field data of the area.

For the actual data processing, we have made a comparison between an interpolated regular grid of $10,000 \times 131$ using a nearest neighbour algorithm and a decimated irregular grid, also of $10,000 \times 131$, totaling $N = 1,310,000$ observation points in both cases. The decimated grid was performed by using the regular grid created in the first case as a guide and by finding the nearest real observation point to this regular grid thus, ensuring that the irregular grid is the lesser deviant possible to conduct the BTTB scheme. In figure \ref{fig:carajas_real_data_decimated_gridline}a we show the result of the interpolation and in \ref{fig:carajas_real_data_decimated_gridline}b the result of the decimation. With $1,310,000$ observation points, it would be necessary $12.49$ Terabytes of RAM to store the full sensitivity matrix with the classical approach. However, taking advantage that the second derivatives of equation \ref{eq:r} are symmetric or skew-symmetric matrices, it is possible to reconstruct the whole sensitivity matrix storing only the first column of each component of equation \ref{eq:Hi}, thus, using only $59.97$ Megabytes, allowing desktop computers being able to process this amount of data.

As this area is very large, different values of the magnetic main field can be considered. 
For this processing, it was considered an approximated mid location of the area (latitude $-6.5^{\circ}$ and longitude $-50.75^{\circ}$) where the declination is $-19.86^{\circ}$ for the IGRF model in 1st january, 2014. The inclination was calculated considering the Geocentric axial dipole model ($tang \, I = 2 \times tan \, \lambda$) and is equal to $12.84^{\circ}$. As the source magnetization is unknown, inclination and declination equals to the main field is being used.

To achieve high efficiency in property estimative of the equivalent sources, the method CGLS for inversion was used, combined with a fast matrix-vector product, only possible because of the BTTB structure of the sensitivity matrix. This fast matrix-vector product was also used for data processing (upward-continuation) in a very efficient way.

Using a equivalent layer at $300$ meters above the ground the predicted data and its residual of the interpolated regular grid are shown in figure \ref{fig:carajas_gz_predito_mag_gridline}. The mean of $0.07979$ nT and the standart deviation of $0.5060$ nT of the residual shows the good result of physical property estimative. It was used 200 iterations of the CGLS method taking $390.80$ seconds with a Intel core i7 7700HQ@2.8GHz processor in single-processing and single-threading modes. Using the same equivalent layer and CGLS configuration the predicted data and its residual of the decimated irregular grid are shown in figure \ref{fig:carajas_gz_predito_mag_decimated}.
With a mean of $0.07348$ nT, standart deviation of $0.3172$ nT and lower residual amplitude compared to the regular grid, we show that the process of decimating the original data, without creating new observation points with interpolation, can be benefical to the method, even if an irregular grid is taken place. It took $385.56$ seconds to complete the estimative.

The convergence analysis for the decimated irregular grid up to $2,000$ CGLS iterations is in figure \ref{fig:convergence_carajas_mag_decimated}, showing good convergence rate and guaranteeing that the irregular grid is not disturbing the method.

In figure \ref{fig:up5000_carajas_decimated_mag} the upward-continuation transformation using the estimated equivalent layer with the decimated grid was made in a horizontal plane at $5,000$ meters and took $2.64$ seconds, showing good results without visible errors or border effects problems and accentuating the long wavelenghts.