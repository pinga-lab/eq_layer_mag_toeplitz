\begin{abstract}

We present a fast equivalent layer method for processing large-scale magnetic data. 
We demonstrate that the sensitivity matrix associated with an equivalent layer
of dipoles assumes a Block-Toeplitz Toeplitz-Block (BTTB) structure for the 
particular case in which observations and equivalent sources are aligned on a 
horizontal and regularly-spaced grid.
The product of a BTTB matrix and an arbitrary vector represents a 2D discrete 
convolution which, in turn, can be efficiently computed via 2D Fast Fourier Transform (2D FFT).
In this case, the matrix-vector product uses only the elements forming the first column
of the BTTB matrix, saving computational time and system memory. 
Our convolutional equivalent layer method uses this fast convolution approach to compute 
the matrix-vector products in the iterative conjugate gradient algorithm with the purpose 
of estimating the physical-property distribution over the equivalent layer and then processing 
large data sets.

Synthetic tests with a mid-size $100 \times 50$ grid of total-field anomaly data
show a decrease of $\approx 10^4$ in floating-point operations and $\approx 25\times$ 
in computation runtime of our method compared to the classical approach of solving
the least-squares normal equations via Cholesky decomposition. 
Better results are obtained for millions of data, showing drastic decreases in RAM usage
and runtime, allowing to perform magnetic data processing of large data sets on regular 
desktop computers. 

PAREI AQUI

Our results also show that, compared to the standard Fourier approach

Synthetic tests simulating data on irregular grids or over undulating observation surfaces  show the robustness of the convolutional equivalent layer in processing magnetic surveys that violate the requirement  that the data be measured on a regular grid and the observation surface be planar.
Test on real magnetic data from Caraj{\'a}s Province, Brazil, with $1,310,000$ observations on an irregular grid, confirms the success of our method, taking $385.56$ seconds to estimate the physical-property distribution over the equivalent layer and $2.64$ seconds for upward-continuing the data.

\end{abstract}