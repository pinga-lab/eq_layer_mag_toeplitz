\section{Introduction}

Throughout the years many authors presented solutions to inverse problems in potential methods using magnetic data. The most common approaches are the works that try to parameterize the geometric source \citep{bott1960use,pilkington1986determination,bhattacharyya1980generalized} and others tried to estimate the susceptibility or the magnetization as the parameter \citep{parker1974inversion,cribb1976application,li19963,liu20173d}. With advances in computation power, the equivalent layer is an inversion technique that is becoming more present in data processing nowadays, despite its use since 1960 in geophysics literature \citep{danes1961structure,bott1967solution,dampney1969}. The equations deductions of the equivalent layer as a solution of the Laplace's equation in the region above the source was first presented by \cite{kellogg1929} and detailed explanations can also be found in \cite{blakely1996}. In magnetic data procesing, some authors explored this technique for calculating the first and second vertical derivatives fields \citep{emilia1973}, reduction to the pole \citep{silva1986,oliveirajr-etal2013,li2014using}, upward/downward continuations \citep{hansen-miyazaki1984,li-oldenburg2010} and total magnetic induction vector components calculation \citep{sun2019constrained}.

Together with the rise in computational processing, some works tried new implementations to increase the efficiency of the equivalent layer. In \cite{leao-silva1989} the authors used a shifting window over the layer, increasing the number of linear systems to be solved, but decreasing the size of it. In \cite{li-oldenburg2010} a method of transforming the full sensitivity matrix into a sparse one by using wavelets was another approach. \cite{oliveirajr-etal2013} also proposed a shifting window method, but instead of directly calculating the equivalent sources physical properties, they estimated the coefficients of a bivariate polynomial funtion representing the sources. An iterative method for gravimetric data was proposed by \cite{siqueira-etal2017}, where the mass distribution over the layer is corrected at each iteration by the residuals of observed and estimated data.

In \cite{takahashi2020convolutional}, the authors combined the fast equivalent source technique presented by \cite{siqueira-etal2017} with the concept of symmetric block-Toeplitz Toeplitz-block (BTTB) matrices to introduce the convolutional equivalent layer for gravimetric data technique. When the sensitivity matrix of the linear system, required to solve the gravimetric equivalent layer, is calculated on a regular spaced grid of dataset with constant height and each equivalent source is exactly beneath each observed data point, the BTTB structure appears. This work showed an decrease in the order of $10^4$ in floating-point operations needed to estimate the equivalent sources.

In potential field methods, the properties of Toeplitz system have been used for downward-continuation \citep{zhang-etal2016} and for 3D gravity-data inversion using a 2D multilayer model \citep{zhang-wong2015}. More recently in \cite{hogue2020tutorial} the authors provided an overview on modeling the gravity and magnetic kernels using the BTTB structures and \cite{renaut2020fast} used for inversion of both gravity and magnetic data to recover sparse subsurface structures.

A wide variety of applications in mathematics and engineering that fall into Toeplitz systems propelled the development of a large variety of  methods for solving them. Direct methods were conceived by \cite{levinson1946} and by \cite{trench1964}. Currently, the iterative method of conjugate gradient is used in most cases, in \cite{chan-jin2007} the authors presented an introduction on the topic for 1D data structures of Toeplitz matrices and also for 2D data structures, which they called block-Toeplitz Toeplitz-block matrices. In both cases, the solving strategy is to embbed the Toeplitz/BTTB matrix into a Circulant/Block-Circulant Circulant-Block matrix, calculate its eigenvalues by a 1D or 2D fast Fourier transform of its first column, respectively and carry the matrix-vector product between kernel and parameters at each iteration of the conjugate gradient method in a very fast manner.

In this work, the convolutional equivalent layer using the block-Toeplitz Toeplitz-block idea, presented in \cite{takahashi2020convolutional}, will be used to solve the linear system required to estimate the physical property that produces a magnetic field on regular grids. Despite the non-existence of a fast equivalent source technique solver as with gravity data, it is possible to achive very fast solutions using a conjugate gradient algorithm combined with the fast Fourier transform. We present a novel method of exploring the symmetric structures of the second derivatives of the inverse of the distance contained in the magnetic kernel, to keep the memory RAM usage to the minimal by using only one equivalent source to carry the calculations of the forward problem. We also show tests of the magnetic convolutional equivalent layer when irregular grids are used. The convergence of the conjugate gradient maintains in an acceptable level even under strong deviations of the data acquisition line. This shows that a fast and robust technique is possible when processing large amounts of magnetic data using the equivalent layer.