%\documentclass[paper,twocolumn,twoside]{geophysics}
%\documentclass[manuscript,noblind]{geophysics}
\documentclass{article}

% Extra packages
\usepackage{amsmath}
\usepackage{algorithm}
\usepackage{bm}
\usepackage[hyphens,spaces]{url}
\usepackage[pdftex,colorlinks=true]{hyperref}
\hypersetup{
	allcolors=black,
}
\usepackage{lipsum}
\usepackage[table]{xcolor}

\begin{document}


{
	\begin{center}
		\begin{tabular}[]{|l|c|c|c|}
			\hline
			\textbf{$N$} & \textbf{$\mathbf{A}$} & \textbf{First columns of matrices $\mathbf{C}_{\boldsymbol{\alpha\beta}}$} & \textbf{$\mathbf{L}$}\\
			\hline 
			$100$ & 0.0763 & 0.0183 & 0.00610\\
			\hline
			$400$ & 1.22 & 0.0744 & 0.0248\\
			\hline
			$2,500$ & 48 & 0.458 & 0.1528\\
			\hline
			$10,000$ & 763 & 1.831 & 0.6104\\
			\hline
			$40,000$ & 12,207 & 7.32 & 2.4416 \\
			\hline
			$250,000$ & 476,837 & 45.768 & 15.3 \\
			\hline
			$500,000$ & 1,907,349 & 91.56 & 30.518 \\
			\hline
			$1,000,000$ & 7,629,395 & 183.096 & 61.035 \\
			\hline
		\end{tabular}
	\end{center} 
\caption{RAM-usage}{This table shows the computer memory usage (in Megabytes) for storing the whole 	$N \times N$ matrix $\mathbf{A}$ (equation 13), the first columns of the BCCB	matrices $\mathbf{C}_{\boldsymbol{\alpha\beta}}$ (equation A-6) (both need 8 bytes per element) and the matrix $\mathbf{L}$ (equation A-18) (16 bytes per element).
\label{tab:RAM-usage}}
}

\end{document}
